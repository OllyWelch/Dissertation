\section{Gauss Sums}
We follow \cite{davenport}. Consider a character $\chi$ to modulus $q$. The \textit{Gauss sum} of $\chi$ is then defined as \begin{equation}
\label{GaussSum}
    \tau(\chi) \coloneqq \sum_{m=1}^{q} \chi(m) e^{2\pi i m/q}. 
\end{equation}
The following Lemma is an important property of Gauss sums. 
\begin{lemma}
For all $n \in \mathbb{N}$ and for all \textbf{primitive} characters $\chi$,
\begin{equation}
\label{GaussRelation}
    \chi(n) \tau(\overline{\chi}) = \sum_{h = 1}^{q} \overline{\chi}(h)e^{2\pi i n h / q} 
\end{equation}
\end{lemma}
\begin{proof}
For $n$ coprime to $q$, this property is clear:
\begin{align}
    \chi(n) \tau(\overline{\chi}) = \sum_{m=1}^{q} \overline{\chi}(m)\chi(n)e^{2 \pi i m / q}
    &= \sum_{m=1}^{q} \overline{\chi}(m n^{-1}) e^{2 \pi i m / q} = \sum_{h=1}^{q} \overline{\chi}(h) e^{2\pi i n h / q}, \nonumber 
\end{align}
where the last step sets $h = m n^{-1}$. If $n$ and $q$ are not coprime, the proof of this fact is rather more delicate, as the left hand side will be zero. It also requires $\chi$ to be primitive. However, proving that the right hand side is zero is not particularly enlightening, so the reader is referred to \cite{davenport}, chapter 9 for details. 
\end{proof}
This Lemma has a useful Corollary. Multiplying both sides of (\ref{GaussRelation}) by their conjugates gives
\begin{align}
    \abs{\chi(n)}^{2}\abs{\tau(\overline{\chi})}^{2} &= \left(\sum_{h_1=1}^{q} \overline{\chi}(h_1) e^{2\pi i n h_1 / q} \right)\overline{\left(\sum_{h_2=1}^{q}\overline{\chi}(h_2) e^{2\pi i n h_2 / q} \right)} \nonumber \\
    &= \sum_{h_1=1}^{q}\sum_{h_2=1}^{q}\overline{\chi}(h_1)\chi(h_2) e^{2\pi i n (h_1 - h_2) / q}. \nonumber
\end{align}
Now sum over all residues modulo $q$. $\abs{\chi(n)}^{2}$ takes the value 1 at precisely $\phi(q)$ values, and is otherwise zero. Meanwhile, the sum of $e^{2\pi i n(h_1 - h_2)/q}$ over all $n$ is zero by symmetry, unless $h_1 \equiv h_2 \ (q)$, in which case it always takes the value 1. Thus, 
\begin{equation}
    \phi(q) \ \abs{\tau(\overline{\chi})}^{2} = q\sum_{h} \abs{\chi(h)}^{2} = q \ \phi(q), \nonumber
\end{equation}
which implies 
\begin{equation}
    \abs{\tau(\chi)} = q^{1/2}
\end{equation}
for all primitive characters $\chi$, upon swapping $\chi$ and $\overline{\chi}$.

