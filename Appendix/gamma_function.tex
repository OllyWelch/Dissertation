\section{Euler's Gamma Function}
For a complex variable $s$ with $\sigma > 0$, Euler's Gamma function is defined as
\begin{equation}
\label{GammaFunction}
\Gamma(s) = \int_{0}^{\infty} x^{s-1} e^{-x} \mathrm{d} x.
\end{equation}
Results in the section on convergence tests imply that the integral converges absolutely and uniformly to a holomorphic function on this region, since
\begin{equation}
    \abs{\Gamma(s)} \leq \int_{0}^{1}x^{\sigma - 1}\mathrm{d}x + C\int_{1}^{\infty}e^{-x/2}\mathrm{d}x < \infty, \nonumber 
\end{equation}
where $C$ is some absolute constant. Moreover, integration by parts gives the key relation
\begin{equation}
\label{GammaRelation}
    s\Gamma(s) = \Gamma(s + 1).
\end{equation}
Since $\Gamma(1) = 1$ (just by computing the integral), it follows by induction that $\Gamma(n) = (n - 1)!$ for positive integers $n$. In this way, the Gamma function is a generalisation of the factorial function. We may also use this relation repeatedly to give the analytic continuation of $\Gamma(s)$. Suppose $\mathfrak{R}(s) \leq 0$, and $s$ is not a non-positive integer. Then there exists $n$ such that $\mathfrak{R}(s + n) > 0$, and by induction of (\ref{GammaRelation}) we have
\begin{equation}
    \Gamma(s) = \frac{\Gamma(s + n)}{s(s + 1)\dots (s + n - 1)}. \nonumber
\end{equation}
Since $s$ is not a non-positive integer, the right hand side is analytic, thus so is the left side. By (\ref{GammaRelation}), we have that Gamma has a simple pole of residue $1$ at $s=0$, and it follows that it also has poles at all negative integers $n$, each with residue $(-1)^{n}/n!$. We now state some useful formulae involving the Gamma function, following \cite{andrews_askey_roy_1999}, and provide sketches of the proofs for brevity.
\begin{definition}
The Beta function is defined for two complex variables $s, s'$ as 
\begin{equation}
B(s, s') = \int_{0}^{1} x^{s-1} (1-x)^{s'-1} \mathrm{d} x. \nonumber
\end{equation}
\end{definition}

\begin{lemma}
\label{BetaGamma}
\begin{equation}
\Gamma(s + s')B(s, s') = \Gamma(s)\Gamma(s'). \nonumber
\end{equation}
\end{lemma}
\begin{proof}
First, note that
\begin{align}
\Gamma(s)\Gamma(s') = \int_{0}^{\infty}\int_{0}^{\infty} x^{s-1}y^{s'-1}e^{-(x+y)} \mathrm{d}x \mathrm{d}y. \nonumber
\end{align}
Make the change of variables $(x, y) = (uz, (1-u)z)$ and the result should follow.
\end{proof}
\begin{theorem}
\label{DuplicationFormula}
(The Duplication Formula)
\begin{equation}
\Gamma(2s) = \pi^{-1/2} 2^{2s-1} \Gamma(s) \Gamma(s + 1/2). \nonumber
\end{equation}
\end{theorem}
\begin{proof}
Change of variables gives that
\begin{align}
B(s, s) = 2^{1-2s}B(1/2, s). \nonumber
\end{align}
Then two uses of Lemma~\ref{BetaGamma} gives
\begin{align}
\Gamma(2s) = \frac{\Gamma(s)^2}{B(s, s)}
= 2^{2s-1}\frac{\Gamma(s)^2}{B(1/2, s)}
&= 2^{2s-1}\frac{\Gamma(s)\Gamma(s + 1/2)}{\Gamma(1/2)}\nonumber \\
&= \pi^{-1/2} 2^{2s-1} \Gamma(s) \Gamma(s + 1/2). \nonumber
\end{align}
\end{proof}
\begin{theorem}
\label{ReflectionFormula}
(The Reflection Formula) For $s \in \mathbb{C}\setminus\mathbb{Z}$, 
\begin{equation}
\Gamma(s)\Gamma(1-s) = \frac{\pi}{\sin{\pi s}}. \nonumber
\end{equation} 
\end{theorem}
\begin{proof}
There is an equivalent definition of $\Gamma(s)$ introduced by Weierstrass. We have
\begin{equation}
\label{WeierstrassProduct}
    \Gamma(s) = s^{-1} e^{-\gamma s} \prod_{n=1}^{\infty}(1 + s/n)^{-1}e^{s/n},
\end{equation}
where $\gamma$ is the Euler-Mascheroni constant, defined as
\begin{equation}
    \gamma = \lim_{n \rightarrow \infty}(\sum_{1}^{n}\frac{1}{k} - \log n). \nonumber
\end{equation}
Since $\sin(s)/s$ is an entire function of order 1 (since $\sin(s) = O(e^{\abs{s}})$) with zeros at non-zero integer multiples of $\pi$, we may use Theorem~\ref{HadamardTheorem} to write it as 
\begin{equation}
    \frac{\sin(s)}{s} = e^{A + Bs} \prod_{n \neq 0}\left(1 - \frac{s}{n\pi} \right) e^{s/n\pi} = e^{A + Bs} \prod_{n=1}^{\infty}\left( 1 - \frac{s^{2}}{n^{2}\pi^{2}} \right), \nonumber
\end{equation}
combining each positive and negative term. Since the left hand side tends to $1$ as $s \rightarrow 0$, it follows that $A = 0$. Moreover, since $\sin(s)/s$ is an even function, we also must have $B = 0$ so that the right hand side is even. We conclude that
\begin{equation}
    \sin(\pi s) = \pi s \prod_{n=1}^{\infty}\left( 1 - \frac{s^{2}}{n^{2}}\right). \nonumber
\end{equation}
Therefore, using (\ref{WeierstrassProduct}) we have
\begin{equation}
    \frac{1}{\Gamma(s)\Gamma(-s)} = -s^{2} \prod_{n=1}^{\infty}\left(1 - \frac{s^{2}}{n^{2}} \right) = -s \  \frac{\sin(\pi s)}{\pi}. \nonumber
\end{equation}
Thus,
\begin{equation}
    \Gamma(s) \Gamma(1 - s) = -s \Gamma(s)\Gamma(-s) = \frac{\pi}{\sin(\pi s)}. \nonumber
\end{equation}
\end{proof}
It follows from this formula that $\Gamma(s)$ is never zero. The right hand side is never zero, so any zero of $\Gamma(s)$ must coincide with a pole of $\Gamma(1-s)$. However, any such pole would be at a non-positive integer, which implies any zero would also have to be an integer. $\Gamma$ is non-zero at the integers, so is therefore non-zero everywhere. The duplication and reflection formulas may be combined to give 
\begin{equation}
    \frac{\Gamma(s/2)}{\Gamma((1-s)/2)} = \pi^{-1/2} 2^{1-s} \cos\frac{\pi s}{2} \Gamma(s). \nonumber 
\end{equation}
Combining this relation with the functional equations for $\zeta(s)$ and $L(s, \chi)$ gives the so-called asymmetric forms of the functional equations, namely
\begin{equation}
\label{asymmetricZeta}
    \zeta(1-s) = 2^{1-s}\pi^{-s}\cos\frac{\pi s}{2}\Gamma(s)\zeta(s),
\end{equation}
and
\begin{equation}
\label{asymmetricL}
    L(1-s, \chi) = \varepsilon(\chi) 2^{1 - s} \pi^{-s} q^{s - 1/2} \cos \frac{\pi(s - a)}{2} \Gamma(s) L(s, \overline{\chi}),
\end{equation}
where $\abs{\varepsilon(\chi)} = 1$. Finally, we have Stirling's approximation to the Gamma function, as given in \cite{davenport}.
\begin{theorem}
\label{StirlingFormula}
(Stirling's Formula) The formula
\begin{equation}
    \log\Gamma(s) = (s - \frac12)\log s - s + \frac12 \log 2\pi + O(\abs{s}^{-1}) \nonumber 
\end{equation}
is valid as $\abs{s} \rightarrow \infty$, where $-\pi + \delta < \arg s < \pi - \delta$ for some fixed $\delta > 0$. 
\end{theorem}