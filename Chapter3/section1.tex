\section{Continuation to the Right Half-Plane}
Our general strategy in analytic continuation of L-functions, and in particular $\zeta(s)$, is to extend them to a slightly larger region first, before proving a powerful functional equation, which allows us to extend each function to $\mathbb{C}$. We first consider the case of L-functions of non-trivial character (which are to some extent easier to extend than $\zeta(s)$ owing to their lack of a singularity at $s=1$) and refer to \cite{ireland_rosen_1990}. First, define 
\begin{equation}
    S(x) = \sum_{n \leq x}\chi(n)
\end{equation}
for a non-trivial character $\chi$ to modulus $q$. For $\sigma > 1$, 
\begin{align}
    \sum_{n=1}^{N}S(n)\left(n^{-s} - (n + 1)^{-s}\right)
    &= \sum_{n \leq N} \chi(n) n^{-s} - S(N)(N+1)^{-s} \rightarrow L(s, \chi) \ \textrm{as} \ N \rightarrow \infty \nonumber,
\end{align}
Note that the secondary term goes to zero since $\abs{S(N)}$ is certainly less than $N$. Therefore:
\begin{align}
\label{LRightPlaneContinuation}
    L(s, \chi) &=  \sum_{n=1}^{\infty}S(n)\left(n^{-s} - (n + 1)^{-s}\right) \nonumber \\
    &= s \sum_{n=1}^{\infty} S(n) \int_{n}^{n+1} x^{-s-1} \mathrm{d} x \nonumber \\
    &= s \int_{1}^{\infty}S(x)x^{-s-1}\mathrm{d}x \quad (\sigma > 1),
\end{align}
where taking $S(x)$ inside the integral is valid as it only changes at discrete values. We now need the following Lemma.
\begin{lemma}
\label{CharacterSumBound}
For a non-trivial character $\chi$ to modulus $q$,
\begin{equation}
    \abs{S(x)} \leq \phi(q) \ \textrm{for all} \ x > 0. \nonumber
\end{equation}
\end{lemma}
Assuming this gives
\begin{align}
    \abs{s \int_{1}^{\infty}S(x) x^{-s-1}\mathrm{d}x} &\leq \abs{s}\int_{1}^{\infty}\abs{S(x)}x^{-\sigma - 1}\mathrm{d}x \nonumber \\
    &\leq \phi(q)\abs{s}\int_{1}^{\infty}x^{-\sigma - 1}\mathrm{d} x < \infty, \quad (\sigma > 0). \nonumber
\end{align}
The lemma therefore implies the following.
\begin{proposition}
For $\chi$ a non-trivial character to modulus $q$, and $S(x)$ defined as above, 
\begin{equation}
    L(s, \chi) = s \int_{1}^{\infty}S(x)x^{-s-1} \mathrm{d} x, \nonumber
\end{equation}
which defines an analytic function on $\sigma > 0$.
\end{proposition}
To prove Lemma~\ref{CharacterSumBound}, for a given $x > 0$, write $x = m q + r$, where $0 \leq r < q$. By the $q$-periodicity of $\chi$,
\begin{equation}
    \sum_{n \leq x} \chi(n) = m\left(\sum_{n=0}^{q-1} \chi(n) \right) + \sum_{n \leq r} \chi(n). \nonumber
\end{equation}
By the first orthogonality relation in Proposition~\ref{OrthogonalityRelations}, the first term is 0. Hence
\begin{equation}
    \abs{S(x)} \leq \sum_{n \leq r} \abs{\chi(n)} \leq \phi(q), \nonumber
\end{equation}
where the last inequality comes from the fact that there are at most $\phi(q)$ non-zero terms, all of modulus 1. \\

Having found an analytic continuation of L-functions of non-trivial character to $\sigma > 1$, we turn to the case of $\zeta(s)$, and by (\ref{LZetaRelation}) the case of L-functions of trivial character. Here, we have the slight issue of the pole at $s = 1$ corresponding to the harmonic series, so we intend to essentially remove it. In this case we have that $S(x) = [x]$, since $\chi(n) = 1$ for all $n$. So, for $\sigma > 1$
\begin{equation}
    \zeta(s) = s\int_{1}^{\infty} [x] x^{-s-1} \mathrm{d} x, \nonumber
\end{equation}
as in (\ref{LRightPlaneContinuation}). Therefore, using the trick of adding and subtracting the same term as in \cite[p.~3]{ivic_2003},
\begin{align}
\label{ZetaRightPlaneContinuation}
    \zeta(s) &= s\int_{1}^{\infty} x^{-s} \mathrm{d} x -  s\int_{1}^{\infty} (x - [x]) x^{-s-1} \mathrm{d} x \nonumber \\
    &= \frac{s}{s - 1} - s\int_{1}^{\infty} (x - [x]) x^{-s-1} \mathrm{d} x \quad (\sigma > 1).
\end{align}
Since $\abs{x - [x]} \leq 1$ for all $x$, we have for $\sigma > 0$
\begin{equation}
    \abs{\int_{1}^{\infty}(x - [x]) x^{-s-1} \mathrm{d} x} \leq \int_{1}^{\infty} x^{-\sigma - 1} \mathrm{d}x < \infty, \nonumber
\end{equation}
so it follows that (\ref{ZetaRightPlaneContinuation}) gives an analytic continuation of $\zeta(s)$ to $\sigma > 0$. Moreover, it shows plainly the value of the residue is $1$ at $s=1$, which we used repeatedly throughout the previous chapter. 