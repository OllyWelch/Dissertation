\section{Continuation to the Right Half-Plane}
Our general strategy in analytic continuation of L-functions, and in particular $\zeta(s)$, is to extend them to a slightly larger region first, before proving a powerful functional equation, which allows us to extend each function to $\mathbb{C}$.
First, consider $\zeta(s)$.
\begin{proposition}
For $\sigma > 0$,
\begin{equation}
\label{righthalfplanecontinuation}
\zeta(s) - \frac{1}{s-1} = \sum_{n=1}^{\infty} \left(\int_{n}^{n+1} (n^{-s} - x^{-s}) \mathrm{d}x \right),
\end{equation}
which gives the analytic continuation to the right half-plane.
\end{proposition}
\begin{proof}
The first part of the proof is to establish equality when $\sigma > 1$. First, note that 
\begin{align}
\sum_{n=1}^{k} \int_{n}^{n+1} x^{-s} \mathrm{d}x &= \frac{1}{1-s} \sum_{n=1}^{k} \left( (n+1)^{1-s} - n^{1-s} \right) \nonumber \\
&= \frac{1}{1-s} \left( -1^{1-s} + (2^{1-s} - 2^{1-s}) + \dots + (k^{1-s} - k^{1-s}) + (k+1)^{1-s} \right) \nonumber \\
&= \frac{1}{1-s} \left((k+1)^{1-s} - 1 \right) \rightarrow \frac{1}{s-1} \hspace{1mm} \textrm{as} \hspace{1mm} k \rightarrow \infty. \nonumber 
\end{align}
Therefore,
\begin{align}
\zeta(s) - \frac{1}{s-1} &= \sum_{n=1}^{\infty} n^{-s} - \sum_{n=1}^{\infty} \int_{n}^{n+1} x^{-s} \mathrm{d} x \nonumber \\
&= \sum_{n=1}^{\infty} \left( n^{-s} -  \int_{n}^{n+1} x^{-s} \mathrm{d} x \right) \nonumber \\
&= \sum_{n=1}^{\infty} \left(\int_{n}^{n+1} (n^{-s} - x^{-s}) \mathrm{d}x \right), \nonumber
\end{align}
since $\int_{n}^{n+1} n^{-s} \mathrm{d} x = n^{-s}$. Since equality has been established for $\sigma > 1$, we now must prove that the right hand side defines an analytic function on the larger region $\sigma > 0$. For $x \in [n, n+1]$,
\begin{equation}
\abs{n^{-s} - x^{-s}} = \abs{s\int_{n}^{x} y^{-1-s} \mathrm{d} y} = \abs{s}\abs{\int_{n}^{x} y^{-1-s} \mathrm{d} y}, \nonumber
\end{equation}
and we may estimate the integral as
\begin{align}
\abs{\int_{n}^{x} y^{-1-s} \mathrm{d} y} &\leq \abs{x-n} n^{-1-\sigma} \leq n^{-1-\sigma}, \nonumber
\end{align}
since $n \leq x \leq n+1$. Therefore,
\begin{equation}
\sum_{n=1}^{\infty} \abs{\int_{n}^{n+1} (n^{-s} - x^{-s}) \mathrm{d}x} \leq \sum_{n=1}^{\infty} \abs{s} n^{-1-\sigma} < \infty, \nonumber
\end{equation}
when $\sigma > 0$. Thus, the right hand side of (\ref{righthalfplanecontinuation}) defines an analytic function on the right half-plane, giving the analytic continuation of $\zeta(s)$ on this region.
\end{proof}
We remark that since we have written $\zeta(s)$ as $(s-1)^{-1}$ plus some analytic function on $\sigma > 0$, it follows that $\zeta(s)$ has residue $1$ at $s=1$, which was assumed in the previous chapter to prove facts about Dirichlet densities, and will be used repeatedly later. L-functions of trivial character have a similar extension to the right half plane by the relation (\ref{LZetaRelation}), so we continue to the case of L-functions of non-trivial character (which are to some extent easier to extend than $\zeta(s)$ owing to their lack of a singularity). First, define $S(x) = \sum_{n \leq x}\chi(n)$ for a non-trivial character $\chi$ to modulus $q$. For $\sigma > 1$, 
\begin{align}
    \sum_{n=1}^{N}S(n)\left(n^{-s} - (n + 1)^{-s}\right)
    &= L_{N}(s, \chi) - S(N)(N+1)^{-s} \rightarrow L(s, \chi) \ \textrm{as} \ N \rightarrow \infty \nonumber,
\end{align}
where $L_{N}(s, \chi)$ denotes the $N$-th partial sum of the series defining $L(s, \chi)$. Therefore:
\begin{align}
    L(s, \chi) &=  \sum_{n=1}^{\infty}S(n)\left(n^{-s} - (n + 1)^{-s}\right) \nonumber \\
    &= s \sum_{n=1}^{\infty} S(n) \int_{n}^{n+1} x^{-s-1} \mathrm{d} x \nonumber \\
    &= s \int_{1}^{\infty}S(x)x^{-s-1}\mathrm{d}x \quad (\sigma > 1). \nonumber
\end{align}
We now need the following Lemma.
\begin{lemma}
\label{CharacterSumBound}
For a non-trivial character $\chi$ to modulus $q$, define $S(x) = \sum_{n \leq x}\chi(n)$. Then $\abs{S(x)} \leq \phi(q)$ for all $x > 0$.
\end{lemma}
Assuming this Lemma gives
\begin{align}
    \abs{s \int_{1}^{\infty}S(x)^{-s-1}\mathrm{d}x} &\leq \abs{s}\int_{1}^{\infty}\abs{S(x)}x^{-\sigma - 1}\mathrm{d}x \nonumber \\
    &\leq \phi(q)\abs{s}\int_{1}^{\infty}x^{-\sigma - 1}\mathrm{d} x < \infty, \quad (\sigma > 0).
\end{align}
Therefore we have proved the following:
\begin{proposition}
For $\chi$ a non-trivial character to modulus $q$, and $S(x)$ defined as above, 
\begin{equation}
    L(s, \chi) = s \int_{1}^{\infty}S(x)x^{-s-1} \mathrm{d} x, \nonumber
\end{equation}
which defines an analytic function on $\sigma > 0$.
\end{proposition}
\begin{proof}
(Lemma~\ref{CharacterSumBound}) \\

For a given $x > 0$, write $x = m q + r$, where $0 \leq r < q$. By the $q$-periodicity of $\chi$,
\begin{equation}
    \sum_{n \leq x} \chi(n) = m\left(\sum_{n=0}^{q-1} \chi(n) \right) + \sum_{n \leq r} \chi(n). \nonumber
\end{equation}
By the first orthogonality relation in Proposition~\ref{OrthogonalityRelations}, the first term is 0. Hence
\begin{equation}
    \abs{S(x)} \leq \sum_{n \leq r} \abs{\chi(n)} \leq \phi(q), \nonumber
\end{equation}
where the last inequality comes from the fact that there are at most $\phi(q)$ non-zero terms, all of modulus 1.
\end{proof}
We have thus extended both $\zeta(s)$ and L-functions of non-trivial character to a slightly larger region. We wish to relate $\zeta(s)$ to $\zeta(1-s)$, and $L(s, \chi)$ to $L(1-s, \overline{\chi})$ respectively. This is done via so-called functional equations, and will complete the analytic continuation of these functions to the complex plane. 