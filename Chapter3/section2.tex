\section{The Key Step in Dirichlet's Theorem}
Having continued L-functions of trivial and non-trivial characters to the right half plane, we now have the tools required to complete the proof of Dirichlet's theorem. We may now study with greater ease the behaviour of L-functions of non-trivial character near $s = 1$, safe in the knowledge that they are analytic here; it therefore suffices to show that they are non-zero at $s = 1$. We follow the argument in \cite{ireland_rosen_1990}. Consider the function
\begin{equation}
    F(s) = \prod_{\chi} L(s, \chi), \nonumber
\end{equation}
where the product is over all Dirichlet characters modulo $q$. We claim that $F(s) \geq 1$ for all real $s > 1$. Recall that 
\begin{equation}
    G(s, \chi) = \sum_{p} \sum_{k=1}^{\infty} k^{-1} \chi(p^{k}) p^{-ks}, \quad (\sigma > 1). \nonumber
\end{equation}
It follows from the orthogonality relations of characters that
\begin{equation}
    \sum_{\chi} G(s, \chi) = \phi(q) \sum_{\substack{k \geq 1 \\ p^{k} \equiv 1 \ (q)}} k^{-1} p^{-ks} \quad (\sigma > 1), \nonumber
\end{equation}
so that $\sum_{\chi}G(s, \chi) \geq 0$ for real $s > 1$ since each term is positive. By taking the exponential map, we have the claim that $F(s) \geq 1$ for all real $s > 1$. Now, consider the behaviour of $F(s)$ as $s \rightarrow 1$. If an L-function of complex character had $L(1, \chi) = 0$, it would follow that $L(1, \overline{\chi}) = \overline{L(1, \chi)} = 0$, so two terms of $F(s)$ are zero. Then, since the only term in the product with a pole at $s = 1$ is the trivial character, and this pole is simple, it follows that $F(1) = 0$. This clearly contradicts $F(s) \geq 1$. Thus $L(1, \chi) \neq 0$ for all complex characters $\chi$. We must use a different approach for real characters as they coincide with their conjugate characters. We assume for a contradiction that $\chi$ is a real character modulo $q$ with $L(1, \chi) = 0$, and consider
\begin{equation}
    \Tilde{F}(s) = \frac{L(s, \chi) L(s, \chi_0)}{L(2s, \chi_0)}, \nonumber
\end{equation}
where $\chi_0$ is the trivial character modulo $q$ as usual. Since the zero of $L(s, \chi)$ cancels the pole of $L(s, \chi_0)$ at $s=1$, the numerator is analytic on $\sigma > 0$, while the denominator is analytic when $\sigma > 1/2$. It follows that $F(s)$ is analytic on $\sigma > 1/2$, and $F(s) \rightarrow 0$ as $s \rightarrow 1$, by virtue of the pole of the denominator. By the Euler products,
\begin{equation}
    \Tilde{F}(s) = \prod_{p} \frac{(1 - \chi_{0}(p)p^{-2s})}{(1 - \chi(p)p^{-s})(1 - \chi_{0}(p)p^{-s})}. \nonumber
\end{equation}
We have $\chi(p) = \pm 1$ if $(p, q) = 1$ since $\chi$ is real, and if $\chi(p) = -1$, the numerator will coincide with the denominator. Moreover, if $\chi(p) = 0$, then $p$ and $q$ are not coprime and each term is $1$. Thus
\begin{equation}
    \Tilde{F}(s) = \prod_{\substack{\chi(p) = 1}} \frac{1 + p^{-s}}{1 - p^{-s}}, \quad (\sigma > 1). \nonumber
\end{equation}
Since
\begin{equation}
    \frac{1 + p^{-s}}{1 - p^{-s}} = (1 + p^{-s})\left( \sum_{k=0}^{\infty} p^{-ks} \right) = 1 + 2p^{-s} + 2p^{-2s} + \dots, \nonumber
\end{equation}
we may write $\Tilde{F}(s)$ as a series of the form $\sum_{n \geq 1} a_n n^{-s}$, where $a_n \geq 0$, convergent for $\sigma > 1$. Note that $a_1 = 1$. Since $\Tilde{F}(s)$ is analytic for $\sigma > 1/2$, we may expand it as a power series around $s = 2$ with radius of convergence at least $3/2$, say $\Tilde{F}(s) = \sum_{m=1}^{\infty} b_m (s - 2)^{m}$. Each $b_m$ is given by $b_m = \Tilde{F}^{(m)}/m!$, and by its representation as an infinite series we have $\Tilde{F}^{(m)}(2) = \sum_{n=1}^{\infty} a_n (-\log n)^{m} n^{-2} = (-1)^{m} c_m$, with $c_m \geq 0$. Therefore we have $\Tilde{F}(s) = \sum_{m=0}^{\infty}d_m (2 - s)^{m}$ with $d_m \geq 0$ and $d_0 = \Tilde{F}(2) = \sum_{n \geq 1} a_n n^{-2} \geq a_1 = 1$. It follows that for real values $1/2 < s < 2$, we have $\Tilde{F}(s) \geq d_0 = 1$, which contradicts $\Tilde{F}(s) \rightarrow 0$ as $s \rightarrow 1/2$. In conclusion, L-functions of any character are non-zero at $s = 1$, which completes the proof of Dirichlet's theorem. \\

Having studied analytic continuation of L-functions to the right half plane and putting this into use, we now turn to analytic continuation on the whole of $\mathbb{C}$, which is done through relating $L(s, \chi)$ to $L(1-s, \overline{\chi})$. The idea is that since $L(s, \chi)$ is analytic on the right half-plane, this relation gives the value of $L(s, \overline{\chi})$ on the left half-plane. Such a relation is known as the functional equation, which we now derive.