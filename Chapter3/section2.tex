\section{Gauss Sums and the Poisson Summation Formula}
The functional equation for L-functions involves a special type of function, known as the Gauss sum. Consider a character $\chi$ to modulus $q$. The \textit{Gauss sum} of $\chi$ is then defined as \begin{equation}
\label{GaussSum}
    \tau(\chi) \coloneqq \sum_{m=1}^{q} \chi(m) e^{2\pi i m/q}. 
\end{equation}
The following Lemma is an important property of Gauss sums. 
\begin{lemma}
For all $n \in \mathbb{N}$ and for all \textbf{primitive} characters $\chi$,
\begin{equation}
\label{GaussRelation}
    \chi(n) \tau(\overline{\chi}) = \sum_{h = 1}^{q} \overline{\chi}(h)e^{2\pi i n h / q} 
\end{equation}
\end{lemma}
\begin{proof}
For $n$ coprime to $q$, this property is clear:
\begin{align}
    \chi(n) \tau(\overline{\chi}) = \sum_{m=1}^{q} \overline{\chi}(m)\chi(n)e^{2 \pi i m / q}
    &= \sum_{m=1}^{q} \overline{\chi}(m n^{-1}) e^{2 \pi i m / q} = \sum_{h=1}^{q} \overline{\chi}(h) e^{2\pi i n h / q}, \nonumber 
\end{align}
where the last step sets $h = m n^{-1}$. If $n$ and $q$ are not coprime, the proof of this fact is rather more delicate, as the left hand side will be zero. It also requires $\chi$ to be primitive. However, proving that the right hand side is zero is not particularly enlightening, so the reader is referred to the appendix for details. 
\end{proof}
This Lemma has a useful Corollary. Multiplying both sides of (\ref{GaussRelation}) by their conjugates gives
\begin{align}
    \abs{\chi(n)}^{2}\abs{\tau(\overline{\chi})}^{2} &= \left(\sum_{h_1=1}^{q} \overline{\chi}(h_1) e^{2\pi i n h_1 / q} \right)\overline{\left(\sum_{h_2=1}^{q}\overline{\chi}(h_2) e^{2\pi i n h_2 / q} \right)} \nonumber \\
    &= \sum_{h_1=1}^{q}\sum_{h_2=1}^{q}\overline{\chi}(h_1)\chi(h_2) e^{2\pi i n (h_1 - h_2) / q}. \nonumber
\end{align}
Now sum over all residues modulo $q$. $\abs{\chi(n)}^{2}$ takes the value 1 at precisely $\phi(q)$ values, and is otherwise zero. Meanwhile, the sum of $e^{2\pi i n(h_1 - h_2)/q}$ over all $n$ is zero by symmetry, unless $h_1 \equiv h_2 \ (q)$, in which case it always takes the value 1. Thus, 
\begin{equation}
    \phi(q) \ \abs{\tau(\overline{\chi})}^{2} = q\sum_{h} \abs{\chi(h)}^{2} = q \ \phi(q), \nonumber
\end{equation}
which implies $\abs{\tau(\chi)} = q^{1/2}$ for all primitive characters $\chi$, upon swapping $\chi$ and $\overline{\chi}$. The importance of this result will be apparent later. \\

A second requirement in the study of the functional equation is the Poisson Summation Formula. We first define a class of functions on which the formula is valid. 
\begin{definition}
A smooth function is \textit{Schwartz} if for every $c \in \mathbb{R}$, $n \in \mathbb{N}\cup\{0\}$:
\begin{equation}
\abs{f^{(n)}(x)} = o(\abs{x}^{c}), \nonumber
\end{equation} 
where $f^{(n)}$ denotes the nth derivative of $f$. In other words, every derivative of the function decays quickly enough to dominate the growth rate of polynomials.
\end{definition}
We now state the theorem.
\begin{theorem}
(Poisson Summation Formula) Let $f : \mathbb{R} \rightarrow \mathbb{C}$ be a Schwartz function, and let $\hat{f}$ be its Fourier transform:
\begin{equation}
\hat{f}(y) = \int_{-\infty}^{\infty} e^{2 \pi i x y} f(x) \mathrm{d} y. \nonumber
\end{equation}
Then
\begin{equation}
\sum_{m \in \mathbb{Z}} f(m) = \sum_{n \in \mathbb{Z}} \hat{f}(n), \nonumber 
\end{equation}
with the sums converging absolutely.
\end{theorem}
\begin{proof}
Define $F(x) = \sum_{m \in \mathbb{Z}} f(x + m)$, and note that since $f$ is Schwartz, we have uniform convergence of the sum to a continuous function. To see this, we can bound each term by some arbitrarily large reciprocal power of $\abs{m}$ multiplied by some large constant, since $f(x+m) = o(\abs{x+m}^c)$ for all real numbers $c$. F is 1-periodic, so consider the Fourier coefficients of $F$:
\begin{align}
\int_{0}^{1} e^{-2 \pi i n x} F(x) \mathrm{d} x &= \int_{0}^{1} e^{-2 \pi i n x} \left( \sum_{m \in \mathbb{Z}} f(x + m) \right) \mathrm{d} x \nonumber \\
&= \sum_{m \in \mathbb{Z}} \int_{0}^{1} e^{-2 \pi i n x} f(x + m) \mathrm{d} x \hspace{1cm} \textrm{(By uniform convergence)} \nonumber \\
&= \sum_{m \in \mathbb{Z}} \int_{m}^{m+1} e^{-2\pi i n(y - m)}f(y) \mathrm{d} y \hspace{1cm} \textrm{(Substitute $x = y - m$)} \nonumber\\
&= \sum_{m \in \mathbb{Z}} \int_{m}^{m+1} e^{-2\pi i ny}f(y) \mathrm{d} y \hspace{1cm} \textrm{($e^{-2 \pi i n(y-m)} = e^{-2 \pi i ny}$)} \nonumber \\
&= \int_{-\infty}^{\infty} e^{-2 \pi i n y} f(y) \mathrm{d} y = \hat{f}(n).  \nonumber
\end{align}
So we have
\begin{equation}
F(x) = \sum_{n \in \mathbb{Z}} \hat{f}(n) e^{i n x}. \nonumber
\end{equation}
Now note that $F(0) = \sum_{m \in \mathbb{Z}} f(m)$ by definition. Therefore
\begin{equation}
\sum_{m \in \mathbb{Z}} f(m) = F(0) = \sum_{n \in \mathbb{Z}} \hat{f}(n). \nonumber
\end{equation}
\end{proof}
This theorem may be applied to $f(n) = e^{-(n + \alpha)^{2} \pi/x}$. Indeed, each derivative is dominated by an exponentially decaying function, so is clearly Schwartz. Thus, we find the Fourier transform: 
\begin{align}
    \hat{f}(n) &= \int_{-\infty}^{\infty} e^{2\pi i n t} f(t) \mathrm{d} t \nonumber \\
    &= \int_{-\infty}^{\infty} e^{2\pi i n t - (t + \alpha)^{2} \frac{\pi}{x}} \mathrm{d} t \nonumber \\
    &= e^{-n^{2} \pi x - 2 \pi i n \alpha} \int_{-\infty}^{\infty} e^{-\frac{\pi}{x}(t - i n x + \alpha)^2} \mathrm{d} t \nonumber \\
    &= \left(\frac{x}{\pi} \right)^{1/2} e^{-n^{2} \pi x - 2\pi i n \alpha}  \int_{-\infty}^{\infty} e^{-v^{2}} \mathrm{d} v \quad \left(\textrm{sub.} \ v = (x / \pi)^{1/2} (t - inx + \alpha) \ \right) \nonumber \\
    &= x^{1/2} e^{-n^{2} \pi x - 2\pi i n \alpha} \nonumber
\end{align}
By the Poisson summation formula, we therefore have the relation
\begin{equation}
\label{ModularRelation}
    \sum_{n=-\infty}^{\infty} e^{-(n + \alpha)^{2} \pi/x} = x^{1/2} \sum_{n=-\infty}^{\infty} e^{-n^{2} \pi x + 2\pi i n \alpha}.  
\end{equation}
We now have what we require to prove the functional equation for L-functions and the Riemann Zeta-function.
