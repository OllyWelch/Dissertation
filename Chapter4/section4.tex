\section{A Zero-Free Region for \texorpdfstring{$\zeta(s)$}{Lg}}
Recall the formula (\ref{ExpicitPsiFormula}). Estimating $\psi(x)$ well, and hence $\pi(x)$, essentially boils down to two things. The first is making the contribution of each individual non-trivial zero, i.e. $\abs{x^{\rho}}$, as small as possible. To ensure this, we aim to find minimum distance $\rho$ must lie from the line $\sigma = 1$ in the critical strip, so that each term is lower order than $x$. The second way of ensuring the contribution of the zeros is minimised is to quantify how many terms there are in the sum. In this section, we will be concerned with the former. Such a minimum distance of each zero from the 1-line is known as a \textit{zero-free region}. The case of trivial and non-trivial characters must be treated separately, owing to the pole at $s=1$ for the trivial character. \\

In any case, our work so far has dealt mainly with primitive characters, so for the trivial character we refer to $\zeta(s)$. Recall (\ref{logZeta}), where
\begin{equation}
    \log\zeta(s) = \sum_{p}\sum_{m=1}^{\infty}m^{-1}p^{-m s} \quad (\sigma > 1). \nonumber
\end{equation}
Since the right hand side is uniformly convergent on $\sigma > 1$, differentiating termwise gives
\begin{equation}
    \frac{\zeta'(s)}{\zeta(s)} = -\sum_{p}\log p \left(\sum_{m=1}^{\infty}p^{-ms} \right) = -\sum_{n=1}^{\infty}\Lambda(n) n^{-s}, \quad (\sigma > 1)\nonumber
\end{equation}
where $\Lambda(n)$ is the von Mangoldt function defined in chapter 2. Therefore
\begin{equation}
\label{RealZetaOverZeta}
    -\mathfrak{R}\frac{\zeta'(s)}{\zeta(s)} =  \sum_{n=1}^{\infty}\Lambda(n) n^{-\sigma} \cos(t \log n) \quad (\sigma > 1).
\end{equation}
Now, consider the inequality, valid for all $\theta$,
\begin{equation}
    0 \leq 2(1 + \cos\theta)^{2} = 3 + 4\cos\theta + \cos 2\theta. \nonumber
\end{equation}
Applied to (\ref{RealZetaOverZeta}), it implies that for all $t$,
\begin{equation}
\label{LinCombZetaInequality}
    3\left(-\frac{\zeta'(\sigma)}{\zeta(\sigma)}\right) + 4\left(-\mathfrak{R}\frac{\zeta'(\sigma + i t)}{\zeta(\sigma + i t)}\right) + \left(-\mathfrak{R}\frac{\zeta'(\sigma + 2i t)}{\zeta(\sigma + 2i t)}\right) \geq 0.
\end{equation}
We now wish to bound each term by above in terms of the non-trivial zeros. Firstly, since $\zeta(s)$ has a simple pole at $s=1$ of residue 1, it follows that for $1 < \sigma \leq 2$:
\begin{equation}
    -\frac{\zeta'(\sigma)}{\zeta(\sigma)} < \frac{1}{\sigma - 1} + A_1, \nonumber
\end{equation}
for some positive constant $A_1$. For complex $s$ with $1 < \sigma \leq 2$, $t \geq 2$, we refer back to the equality (\ref{ZetaPartialFraction}). Since Stirling's formula implies $\log\Gamma(s) \sim s\log s$, we have
\begin{equation}
    \abs{\frac12\frac{\Gamma'(s/2 + 1)}{\Gamma(s/2 + 1)}} < A_2\log t, \nonumber
\end{equation}
for a positive constant $A_2$. Since all other terms in (\ref{ZetaPartialFraction}) except the sum are bounded,
\begin{equation}
\label{Chapter4Inequality1}
    -\mathfrak{R}\frac{\zeta'(s)}{\zeta(s)} < A_2\log t - \sum_{\rho} \mathfrak{R}\left( \frac{1}{s-\rho} + \frac{1}{\rho} \right).
\end{equation}
Furthermore, since
\begin{equation}
    \mathfrak{R}\frac{1}{s - \rho} = \frac{\sigma - \beta}{\abs{s-\rho}^{2}}, \quad \textrm{and} \quad \mathfrak{R}\frac{1}{\rho} = \frac{\beta}{\abs{\rho}^{2}}, \nonumber
\end{equation}
the sum over zeros is strictly positive. We can therefore throw away any number of terms in (\ref{Chapter4Inequality1}), and the inequality will still hold. Choose $t$ such that $t$ coincides with the imaginary part of a non-trivial zero, say $\rho_1 = \beta + i\gamma$. Then we certainly have
\begin{equation}
    -\mathfrak{R}\frac{\zeta'(\sigma + 2 i t)}{\zeta(\sigma + 2 i t)} < A_{2}\log t, \nonumber
\end{equation}
and 
\begin{equation}
     -\mathfrak{R}\frac{\zeta'(\sigma + i t)}{\zeta(\sigma + i t)} < A_{2}\log t - \frac{1}{\sigma - \beta},
\end{equation}
where the second inequality throws away all but the $(s-\rho)^{-1}$ term in the sum corresponding to the zero $\rho_1$. Putting all of these inequalities into (\ref{LinCombZetaInequality}), we obtain
\begin{equation}
    \frac{4}{\sigma - \beta} < \frac{3}{\sigma - 1} + A \log t, \nonumber
\end{equation}
where $A$ is a positive constant. Let $\sigma = 1 + \delta / \log t$. Then, sparing tedious algebraic details, we have
\begin{equation}
   \beta < 1 - \frac{\delta(1 - A\delta)}{(3 + A \delta) \log t}, \nonumber
\end{equation}
which, upon choosing $\delta$ small enough in relation to $A$, gives an absolute constant $c$ such that
\begin{equation}
    \beta < 1 - \frac{c}{\log t}. \nonumber
\end{equation}
Since $\beta + i \gamma$ was an arbitrary zero, we may therefore find a $c$ for any $t \geq 2$ such that any zero satisfies this property. It follows from the functional equation that if $\rho$ is a non-trivial zero of $\zeta(s)$, then so is $\overline{\rho}$. Therefore the above region holds upon replacing $t$ by $\abs{t}$. Hence, we have found an explicit zero free region for $\zeta(s)$, and proved the following.
\begin{theorem}
For all $s=\sigma + it$ satisfying
\begin{equation}
    \sigma \geq 1 - \frac{c}{\log \abs{t}}, \quad \abs{t} \geq 2, \nonumber
\end{equation}
we have $\zeta(s) \neq 0$. 
\end{theorem}
In order to extend this theorem to all $t$, we must prove that there are no zeros arbitrarily close to $\sigma = 1$ where $\abs{t} < 2$. This is implied by the statement that $\zeta(1 + i t) \neq 0$ for all $t$. To show this, note that an identical argument as before involving the (double) infinite sum representation of $\log \zeta(s)$ in (\ref{logZeta}) gives
\begin{equation}
    3\log\zeta(\sigma) + 4\mathfrak{R}\log\zeta(\sigma + it) + \mathfrak{R} \log \zeta(\sigma + 2it) \geq 0, \nonumber
\end{equation}
so that exponentiation implies
\begin{equation}
\label{ZetaPowerRelation}
    \abs{\zeta^{3}(\sigma)\zeta^{4}(\sigma + it)\zeta(\sigma + 2it)} \geq 1,
\end{equation}
all of this when $\sigma > 1$. Suppose $\zeta(1 + it) = 0$. As $\sigma \rightarrow 1$, $\zeta(\sigma) \sim (\sigma - 1)^{-1}$, while $\zeta(\sigma + it) \sim (\sigma - 1)$ (by assumption), so that the left hand side of  (\ref{ZetaPowerRelation}) goes to zero. This clearly contradicts (\ref{ZetaPowerRelation}), so we conclude that $\zeta(1 + it)$ is non-zero. Thus, we may extend the zero free region to all $s$ satisfying
\begin{equation}
    \sigma > 1 - \frac{c}{\log(\abs{t} + 2)}. \nonumber
\end{equation}