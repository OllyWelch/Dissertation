To summarise, we have studied the frequency of prime numbers in detail, particularly  those in arithmetic progressions. We began by using elementary properties of Dirichlet characters and L-functions around $s = 1$ to prove Dirichlet's theorem, the fact that the arithmetic progression $qn + a$, with $a$ and $q$ coprime contains infinitely many primes. \\

Following this, we found the analytic continuation of L-functions to the whole complex plane, using methods discovered by Riemann. Using this analytic continuation, we wrote the logarithmic derivative of L-functions as infinite sums in terms of their non-trivial zeros, which exploited the Hadamard product formula. In addition to elementary bounds on L-functions, this yielded information on the position of zeros in the critical strip, as well as their frequency. Armed with this information, it was then possible to use a clever contour integration trick on the logarithmic derivative of L-functions, which related the non-trivial zeros of L-functions to the function $\psi(x, \chi)$ through Euler's product formula. \\

We first followed this procedure in the special case of $\zeta(s)$, which gave us an asymptotic formula for $\psi(x)$ and hence for $\pi(x)$ - the number of primes less than a given value - completing our proof of the prime number theorem. We proceeded to the more general case of L-functions, which gave an asymptotic formula for $\psi(x, \chi)$ with the caveat of a term involving a problematic real zero of a particular L-function. Thus, we dealt with the problem of possible real zeros near $s = 1$ through Siegel's theorem, which allowed us to complete the formula for $\psi(x, \chi)$. This then gave us the prime number theorem for arithmetic progressions: the number of primes less than a given value $x$ in the arithmetic progression $qn + a$ is asymptotically $\textrm{Li}(x)/\phi(q)$ (as long as $q$ and $a$ are coprime). \\

Following the proof of this significant result, it was important to note that it didn't tell us the whole story. Firstly, it didn't give us the best possible error term, and we showed that the assumption of GRH gives us a much stronger result. We also noted that this formula was only asymptotic: its more local behaviour revealed certain biases towards certain arithmetic progressions, namely those where $a$ was non-square modulo $q$. Furthermore, we noticed how these biases become even more wild when considering sequences of consecutive primes. \\

The main takeaway from this report is that although the primes seem well-behaved in the long run, if you zoom in enough to smaller scales, the behaviour becomes more and more bizarre. Let us return to Quote~\ref{EulerQuote}: was Euler right in what he said? Perhaps we have found some order on a large scale, but we have seen that delving further and further into the primes only adds more and more mystery; whether this mystery will ever be solved is uncertain. Consequently, there is no doubt that the study of primes will be at the forefront of mathematical research for many years to come.