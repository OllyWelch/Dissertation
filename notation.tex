\makenomenclature
\renewcommand{\nomname}{Notation}

\renewcommand{\nompreamble}{The following notation will be used throughout the report.}

\nomenclature[21]{$f(x) = O\left(g(x)\right)$}{means $\exists \ C, \ x_0 > 0$ s.t. $\forall \ x > x_0$, $\abs{f(x)} \leq Cg(x)$.}

\nomenclature[22]{$f(x) = g(x) + O\left(h(x)\right)$}{means $\abs{f(x) - g(x)} = O\left(h(x)\right)$.}

\nomenclature[23]{$f(x) \sim g(x)$}{means $f(x)/g(x) \rightarrow 1$ as $x \rightarrow \infty$.}

\nomenclature[20]{$f(x) \ll g(x)$}{means $\exists \ C>0$ such that $f(x) \leq Cg(x)$.}

\nomenclature[15]{$a \equiv b \ (q)$}{means $a$ is congruent to $b$ modulo $q$.}

\nomenclature[16]{$\phi(q)$}{Euler's totient function evaluated at $q$.}

\nomenclature[11]{$\mathfrak{R}(z)$}{The real part of a complex number $z$.}

\nomenclature[12]{$\mathfrak{I}(z)$}{The imaginary part of a complex number $z$.}

\nomenclature[13]{$\overline{z}$}{The complex conjugate of a complex number $z$.}

\nomenclature[14]{$\overline{n}$}{The residue class of an integer $n$ to a given modulus.}

\nomenclature[00]{$\mathbb{N}$}{The natural numbers, $\{1, 2, 3, \dots\}$}

\nomenclature[01]{$\mathbb{Z}$}{The integers, $\{0, \pm1, \pm2, \dots \}$}

\nomenclature[02]{$\mathbb{R}$}{The set of real numbers.}

\nomenclature[03]{$\mathbb{C}$}{The set of complex numbers.}

\nomenclature[04]{$R^{\times}$}{The group of units of a ring $R$.}

\printnomenclature