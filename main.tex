% !TEX TS-program = pdflatex
% !TEX encoding = UTF-8 Unicode

% This is a simple template for a LaTeX document using the "article" class.
% See "book", "report", "letter" for other types of document.

\documentclass[11pt]{report} % use larger type; default would be 10pt

\usepackage[utf8]{inputenc} % set input encoding (not needed with XeLaTeX)

%%% Examples of Article customizations
% These packages are optional, depending whether you want the features they provide.
% See the LaTeX Companion or other references for full information.

%%% PAGE DIMENSIONS
\usepackage{geometry} % to change the page dimensions
\geometry{a4paper} % or letterpaper (US) or a5paper or....
% \geometry{margin=2in} % for example, change the margins to 2 inches all round
% \geometry{landscape} % set up the page for landscape
%   read geometry.pdf for detailed page layout information

\usepackage{graphicx} % support the \includegraphics command and options

% \usepackage[parfill]{parskip} % Activate to begin paragraphs with an empty line rather than an indent

%%% PACKAGES
\usepackage{booktabs} % for much better looking tables
\usepackage{array} % for better arrays (eg matrices) in maths
\usepackage{paralist} % very flexible & customisable lists (eg. enumerate/itemize, etc.)
\usepackage{verbatim} % adds environment for commenting out blocks of text & for better verbatim
\usepackage{subfig} % make it possible to include more than one captioned figure/table in a single float
% These packages are all incorporated in the memoir class to one degree or another...

%%% HEADERS & FOOTERS
\usepackage{fancyhdr} % This should be set AFTER setting up the page geometry
\pagestyle{fancy} % options: empty , plain , fancy
\renewcommand{\headrulewidth}{0pt} % customise the layout...
\lhead{}\chead{}\rhead{}
\lfoot{}\cfoot{\thepage}\rfoot{}

%%% SECTION TITLE APPEARANCE
%\usepackage{sectsty}
%\allsectionsfont{\sffamily\mdseries\upshape} % (See the fntguide.pdf for font help)
% (This matches ConTeXt defaults)

%%% ToC (table of contents) APPEARANCE
\usepackage[nottoc,notlof,notlot]{tocbibind} % Put the bibliography in the ToC
\usepackage[titles,subfigure]{tocloft} % Alter the style of the Table of Contents
\renewcommand{\cftsecfont}{\rmfamily\mdseries\upshape}
\renewcommand{\cftsecpagefont}{\rmfamily\mdseries\upshape} % No bold!

%%% END Article customizations

\usepackage{amssymb}
\usepackage{amsthm}
\usepackage{amsmath}
\usepackage{mathtools}
\usepackage{float}
\usepackage{appendix}

\newcommand{\pdiff}[2]{\frac{\partial #1}{\partial #2}}
\newcommand{\pddiff}[2]{\frac{\partial^2 #1}{\partial #2^2}}
\numberwithin{equation}{chapter}


%%% The "real" document content comes below...

\title{The Distribution of Prime Numbers in Arithmetic Progressions}
\author{Olly Welch}
%\date{} % Activate to display a given date or no date (if empty),
         % otherwise the current date is printed 

\begin{document}
\maketitle

\DeclarePairedDelimiter\abs{\lvert}{\rvert}
\DeclarePairedDelimiter\norm{\lVert}{\rVert}
\makeatletter
\let\oldabs\abs
\def\abs{\@ifstar{\oldabs}{\oldabs*}}
\let\oldnorm\norm
\def\norm{\@ifstar{\oldnorm}{\oldnorm*}}
\makeatother

\newtheorem{theorem}{Theorem}[chapter]
\newtheorem{corollary}[theorem]{Corollary}
\newtheorem{lemma}[theorem]{Lemma}
\newtheorem{proposition}[theorem]{Proposition}
\theoremstyle{definition}
\newtheorem{definition}[theorem]{Definition}
\theoremstyle{remark}
\newtheorem*{remark}{Remark}

%TABLE OF CONTENTS
\tableofcontents

%NEW CHAPTER INTRODUCTION
\chapter{Introduction}

%NEW CHAPTER DIRICHLET L-FUNCTIONS
\chapter{Dirichlet L-Functions}
\section{Definition and Connection to Primes}
To study the distribution of primes, we must first define a very important class of function, known as Dirichlet L-functions. These are defined for a complex variable, whose standard notation is $s = \sigma + i t$. Unless otherwise stated, we follow \cite{ireland_rosen_1990} here.
\begin{definition}
\label{LFunctionDefinition}
For $\sigma > 1$, we define the L-function
\begin{equation}
    L(s, \chi) \coloneqq \sum_{n=1}^{\infty} \chi(n) n^{-s}, \nonumber
\end{equation}
where $\chi(n)$ is a so-called Dirichlet character of modulus $q \in \mathbb{N}$.
\end{definition}
The Dirichlet characters modulo $q$ are another special class of function which are interesting in their own right. They are defined as follows.
\begin{definition}
\label{DirichletCharacterDefinition}
For a given $q \in \mathbb{N}$, a Dirichlet character $\chi$ of modulus $q$ is a complex function from the integers satisfying:
\begin{itemize}
    \item $\chi(n + q) = \chi(n)$ for all $n \in \mathbb{Z}$ ($q$-periodicity).
    \item $\chi(m n) = \chi(m) \chi(n)$ (Multiplicativity).
    \item $\chi(n) \neq 0$ if and only if $n$ is coprime to $q$.
\end{itemize}
\end{definition}
For the time being, we will not concern ourselves with why these characters are so important, but simply use their properties to exhibit the connection of L-functions to the primes. L-functions are generalisations of the famous \textit{Riemann Zeta-Function},
\begin{equation}
    \zeta(s) \coloneqq \sum_{n=1}^{\infty} n^{-s}, \quad (\sigma > 1).
\end{equation}
First note that $\zeta(s)$ is uniformly convergent to an analytic function on $\sigma > 1$. For we can bound the absolute value of each term in the infinite sum by $n^{-1 - \varepsilon}$, for some $\varepsilon > 0$. Then since
\begin{equation}
    \int_{1}^{\infty} x^{-1 - \varepsilon} \mathrm{d}x < \infty, \nonumber 
\end{equation}
the integral test implies convergence of $\sum_{n} n^{-1-\varepsilon}$. We may therefore bound the absolute value of $\zeta(s)$ on $\sigma > 1$ by this convergent sum independent of $s$, so by Weierstrass's M-test we have uniform convergence. It is then a consequence of Proposition~\ref{UniformLimitHolo} (see appendix - we will use this result implicitly from now) that $\zeta(s)$ is holomorphic on this region. L-functions are also holomorphic on this region: we may use the same bound as for $\zeta(s)$, following easily the fact that $\abs{\chi(n)} \leq 1$ for all integers $n$, which will be shown in the next section. We may now elucidate the connection of these functions to the primes via the following theorem.
\begin{theorem}\label{sumsAndProducts}
Let $f : \mathbb{N} \rightarrow \mathbb{C}$ be a multiplicative function. That is, for every coprime $m, n \in \mathbb{N}$, we have $f(mn) = f(m)f(n)$. Suppose that $\sum_{n=1}^{\infty}\abs{f(n)} < \infty$. Then
\begin{equation}
\sum_{n=1}^{\infty} f(n) = \prod_{p \hspace{1mm} \mathrm{prime}} \left(\sum_{n=0}^{\infty} f(p^{n}) \right), \nonumber
\end{equation} 
where the infinite product is over every prime number $p$.
\end{theorem}
\begin{proof}
We refer to \cite{ivic_2003}. Using the fact that $\sum_{n=1}^{\infty}\abs{f(n)} < \infty$, we may expand the brackets of terms in the product over finitely many primes. This gives
\begin{equation}
\prod_{p \leq x} \left(1 + f(p) + \dots \right) = \sum_{n \in S_x} f(n). \nonumber
\end{equation} 
where $S_x$ denotes the subset of natural numbers whose prime factors are all less than or equal to $x$. This follows from the multiplicativity of $f$ and the fundamental theorem of arithmetic, so that each term in the expansion of the finite product is covered once and only once on the right hand side. Then let 
\begin{equation}
\sum_{n \in S_x} f(n) = \sum_{n \leq x} f(n) + R(x).\nonumber
\end{equation} 
Here, $R(x)$ is the sum of $f(n)$ for all the numbers greater than $x$ with prime factors less than or equal to $x$, so that each term in the left hand side is covered exactly once. Therefore
\begin{equation}
\abs{R(x)} \leq \sum_{n > x} \abs{f(n)} \rightarrow 0 \hspace{1mm} \textrm{as} \hspace{1mm} x \rightarrow \infty, \nonumber
\end{equation} 
since the tails of a convergent sum tend to zero. Hence, we let $x \rightarrow \infty$, to give
\begin{equation}
\prod_{p \hspace{1mm} \textrm{prime}} \left(1 + f(p) + \dots \right) = \sum_{n =1}^{\infty} f(n). \nonumber
\end{equation} 
\end{proof}
In the case of L-functions, the hypothesis of the theorem is satisfied: the terms in the series are multiplicative (owing to the properties of Dirichlet characters) and absolutely convergent on $\sigma > 1$. Therefore we have
\begin{align}
    L(s, \chi) &= \prod_{p \ \mathrm{prime}} \left(\sum_{n=0}^{\infty} \chi(p^{n})p^{-ns} \right) \nonumber \\
    &= \prod_{p \ \mathrm{prime}} \left(\frac{1}{1 - \chi(p)p^{-s}} \right), \quad (\sigma > 1)
\end{align}
where the last step applies the formula for the sum of a geometric series, which is justified as the absolute value of each term is less than 1. In the case of the Zeta-function, we recover the famous product formula first discovered by Euler:
\begin{equation}
    \label{EulerEquation}
    \zeta(s) = \prod_{p \ \mathrm{prime}} \left(\frac{1}{1 - p^{-s}} \right), \quad (\sigma > 1) 
\end{equation}
These products are the key in answering questions on the distribution of prime numbers. One immediate consequence of (\ref{EulerEquation}) is that there are infinitely many primes: as $s \rightarrow 1$, the Zeta-function becomes the harmonic series and diverges. The infinite product therefore never terminates, as otherwise its value would be finite, implying the infinitude of primes. We will see that studying the behaviour of L-functions at $s=1$ will allow us to prove a similar result for primes in arithmetic progressions. Before doing so, we need more knowledge on the properties of Dirichlet characters.\\
\section{Properties of Dirichlet Characters}
The easiest way to construct the Dirichlet characters modulo $q$ is algebraically. Consider the multiplicative group of units of the integers modulo $q$, $G = (\mathbb{Z}/q\mathbb{Z})^{\times}$. Then, if $\chi'$ is a group homomorphism to $\mathbb{C}^{\times}$, we may construct a character $\chi$ by setting $\chi(n)=0$ for $n$ not coprime to $q$, and $\chi(n) = \chi'(\overline{n})$ otherwise, where $\overline{n}$ is the residue class of $n$ modulo $q$. Clearly all three properties of a Dirichlet character hold for $\chi$, with multiplicativity coming from the homomorphism property of $\chi'$. \\

Conversely, suppose $\chi$ is a Dirichlet character. From the third property in Definition~\ref{DirichletCharacterDefinition}, we must have that $\chi$ is only non-zero at integers coprime to $q$: precisely those whose residue classes are in $G$. Furthermore, $q$-periodicity means that $\chi$ must be single-valued on each residue class modulo $q$, while multiplicativity guarantees that $\chi$ restricted to $G$ is a homomorphism from $G$ to $\mathbb{C}^{\times}$. Therefore, the Dirichlet characters modulo $q$ are precisely those following this construction. \\

First, note that the function $\chi_{0}(n) = 1$ for all $n$ coprime to $q$, $\chi_{0}(n) = 0$ otherwise is always a character - we call this the \textit{trivial character}. Furthermore, since group homomorphisms always map identity to identity, we have $\chi(1) = 1$ for all characters $\chi$. Consequently, all non-zero values of $\chi$ must be $\phi(q)$-th roots of unity, where $\phi$ denotes Euler's totient function. Indeed, since for all $n$ coprime to $q$, $n^{\phi(q)} \equiv 1$ $(q)$, we have $1 = \chi(1) = \chi(n^{\phi(q)}) = \chi(n)^{\phi(q)}$, using the periodic and multiplicative properties. It follows that characters always have absolute value 1 on values $n$ coprime to $q$, so that $1 = \abs{\chi(n)}^{2} = \chi(n)\overline{\chi(n)}$. Thus, the inverse to each character is its complex conjugate. Note that the complex conjugate also defines a character, denoted $\overline{\chi}$, since for $m, n$ coprime to $q$,
\begin{equation}
    \overline{\chi}(m n) = \overline{\chi(m n)} = \overline{\chi(m)} \ \overline{\chi(n)} = \overline{\chi}(m)\overline{\chi}(n). \nonumber
\end{equation}
We now bring in some facts from representation theory, exposed in \cite{serre-reps}. Since the Dirichlet characters restricted to $G$ are homomorphisms to $\mathbb{C}^{\times}$, they are one-dimensional (and by default irreducible) representations of this group. Conversely, since $G$ is abelian, all such irreducible representations are one-dimensional, hence giving Dirichlet characters. Thus, the group characters (the trace of the irreducible representations) coincide with the Dirichlet characters restricted to $G$. Therefore facts about group characters also hold for Dirichlet characters.
\begin{proposition}
\label{isomorphism}
The number of Dirichlet characters modulo $q$ is $\phi(q)$.
\end{proposition}
This follows from the fact that the number of group characters is the number of conjugacy classes of the group. Since $G$ is abelian, this number is the size of the group, $\phi(q)$. 
Furthermore, it is a fact that characters are ``orthogonal", which we may state as following.
\begin{proposition}
\label{OrthogonalityRelations}
If $\chi$ and $\psi$ are Dirichlet characters modulo $q$, and $a, b \in \mathbb{Z}$, then
\begin{enumerate}
    \item $\sum_{n=0}^{q-1} \chi(n)\overline{\psi(n)} = \phi(q)$ if $\chi=\psi$, $0$ otherwise.
    \item $\sum_{\chi}\chi(a)\overline{\chi(b)} = \phi(q)$ if $a=b$, $0$ otherwise,
\end{enumerate}
where the sum is over all Dirichlet characters $\chi$ modulo $q$.
\end{proposition}
We also note the important notion of \textit{induced} and \textit{primitive} Dirichlet characters, following \cite{heath-brown_2005}: a character $\psi$ to modulus $q$ is said to be \textit{induced} by another character $\chi$ to modulus $m < q$ if $\chi(n) = \psi(n)$ for all $n$ coprime to $q$. For example, consider the character to modulus 4, given as
\begin{center}
    \begin{tabular}{c|c c c c}
        $n$ &  1 & 2 & 3 & 4\\
        \hline
        $\chi(n)$ & 1 & 0 & -1 & 0
    \end{tabular}
\end{center}
We may then produce an induced character $\psi$ to modulus 8, given as:
\begin{center}
    \begin{tabular}{c|c c c c c c c c}
        $n$ & 1 & 2 & 3 & 4 & 5 & 6 & 7 & 8\\
        \hline
        $\psi(n)$ & 1 & 0 & -1 & 0 & 1 & 0 & -1 & 0
    \end{tabular}
\end{center}
If a character is not induced by any other character of lower modulus, then it is said to be \textit{primitive}. Thus, in this example $\chi$ is in fact primitive, while $\psi$ is induced. L-functions of an induced character are related to the L-function of the corresponding primitive character as follows. Suppose $\chi$ of modulus $q$ is induced by the primitive character $\chi_1$. Then for $\sigma > 1$, by the product formula,
\begin{equation}
\label{InducedCharacterRelation}
    L(s, \chi) = \prod_{p \ \textrm{prime}}(1 - \chi(p)p^{-s}) = L(s, \chi_1) \prod_{p \rvert q}(1 - \chi_{1}(p)p^{-s}).
\end{equation}
In particular, notice that $\zeta(s)$ is the special case when $q = 1$ and the character is trivial. All subsequent trivial characters are induced by this character, so if $\chi_0$ is the trivial character modulo $q$,
\begin{equation}
\label{LZetaRelation}
    L(s, \chi_{0}) = \zeta(s) \prod_{p \rvert q}(1 - p^{-s}) \quad (\sigma > 1). 
\end{equation}
Now, using the properties of Dirichlet characters, we may sum L-functions over all characters modulo $q$, which will yield interesting results concerning primes in arithmetic progressions. The first of which is \textit{Dirichlet's Theorem}, which we study next.
\section{Dirichlet's Theorem}
This section aims to outline the proof of Dirichlet's theorem as in \cite{ireland_rosen_1990} - the fact that there are infinitely many primes congruent to each $a$ coprime to any modulus $q$, with ``equal density" among each of these residue classes. In order to state this idea rigorously, we must first make an important definition. 
\begin{definition}
(Dirichlet Density) Given a set $\mathcal{P}$ of positive prime numbers, we say that $\mathcal{P}$ has \textit{Dirichlet density} if
\begin{equation}
   \lim_{s \rightarrow 1}\frac{\sum_{p \in \mathcal{P}}p^{-s}}{\log (s - 1)^{-1}}. \nonumber
\end{equation}
exists. We denote the value of the limit by $d(\mathcal{P})$, named the Dirichlet density of $\mathcal{P}$.
\end{definition}
It is clear from the definition that if a set of primes has non-zero Dirichlet density, then the set contains infinitely many primes. Moreover, if $\mathcal{P}$ can be written as the disjoint union of two sets, then the Dirichlet density of $\mathcal{P}$ will be equal to the sum of the Dirichlet densities of the disjoint components. It is a fact that if $\mathcal{P}$ contains all but finitely many positive primes, then its Dirichlet density is 1. Indeed, from (\ref{EulerEquation}), we may write the logarithm of $\zeta(s)$ (for real values $s > 1$) as 
\begin{align}
\label{logZeta}
    \log \zeta(s) &= \sum_{p} -\log(1 - p^{-s}) \nonumber \\
    &= \sum_{p} \sum_{m=1}^{\infty} m^{-1} p^{-ms} \nonumber \\
    &= \sum_{p} p^{-s} + \sum_{p} \sum_{m=2}^{\infty} m^{-1} p^{-ms}, 
\end{align}
where the second step comes from the Taylor expansion of $\log(1 - x)$. Furthermore, we have that 
\begin{align}
    \sum_{p}\sum_{m=2}^{\infty}m^{-1}p^{-ms} &< \sum_{p}\sum_{m=2}^{\infty}p^{-ms}
    = \sum_{p}p^{-2s}(1 - p^{-s})^{-1} \nonumber \\
    &\leq (1 - 2^{-s})^{-1}\sum_{p}p^{-2s} < 2\zeta(2), \nonumber
\end{align}
so $\log \zeta(s) = \sum_{p} p^{-s} + R(s)$ where $R$ is a function which remains bounded as $s \rightarrow 1$. We will also see that $\zeta(s)$ has residue 1 at $s = 1$, so in particular $\lim_{s \rightarrow 1^{+}}(s-1)\zeta(s) = 1$. Letting $\rho(s) = (s-1)\zeta(s)$, we have
\begin{equation}
    \frac{\log \zeta(s)}{\log(s - 1)^{-1}} = 1 + \frac{\log\rho(s)}{\log(s - 1)^{-1}}. \nonumber
\end{equation}
Since $\rho(s) \rightarrow 1$ as $s \rightarrow 1$, its logarithm goes to zero. In particular
\begin{align}
    \lim_{s \rightarrow 1^{+}}\frac{\sum_{p}p^{-s}}{\log(s - 1)^{-1}} = \lim_{s \rightarrow 1^{+}}\frac{\log\zeta(s)}{\log(s-1)^{-1}} = 1.
\end{align}
Thus, the Dirichlet density of all the primes with the exception of finitely many is 1. In summary, the Dirichlet density is a number between 0 and 1 which measures (in a sense) the ``proportion" of primes which lie in a set. We may now use this to state Dirichlet's theorem.
\begin{theorem}
(Dirichlet's Theorem) Suppose $a$ and $q$ are coprime integers. Let $\mathcal{P}(a; q)$ denote the set of all primes $p$ such that $p \equiv a \ (q)$. Then
\begin{equation}
    d\left(\mathcal{P}(a; q)\right) = 1/\phi(q). \nonumber
\end{equation}
In particular, the primes have equal Dirichlet density among residue classes coprime to $q$, and there are infinitely many primes in each such residue class.
\end{theorem}
This may be proved in a similar manner to proving the Dirichlet density of the set of all primes is 1. Rather than taking the logarithm of $\zeta(s)$, we wish to take the logarithm of $L(s, \chi)$. However, in this case there is the technicality of branch cuts to consider: even when restricting to real $s > 1$, the L-function may still take on complex values. Therefore we must proceed in a more subtle manner. Consider 
\begin{equation}
    G(s, \chi) = \sum_{p}\sum_{k=1}^{\infty} k^{-1}\chi(p^{k})p^{-ks}. \nonumber
\end{equation}
Notice that this is similar to the infinite series definining $\log\zeta(s)$. Also, since the absolute value of each term of $G$ is bounded by $p^{-ks}$, $G$ is bounded by $\zeta(s)$ which is uniformly convergent for real $s > 1$, so the same is true for $G$. Moreover, taking the exponential map, we see that
\begin{align}
    \exp(G(s, \chi)) &= \prod_{p} \exp\left(\sum_{k=1}^{\infty}k^{-1}(\chi(p)p^{-s})^{k}\right) \nonumber \\
    &= \prod_{p} (1 - \chi(p)p^{-s})^{-1} = L(s, \chi), \nonumber
\end{align}
where the penultimate step comes from the Taylor series of $\log(1 - \chi(p)p^{-s})$, justified since $\abs{\chi(p)p^{-s}} < 1$. Thus, the infinite series $G$ gives a definition of $\log L(s, \chi)$ without the need to worry about branch cuts until later. We therefore proceed working directly with $G$. In a similar manner to before, we have that 
\begin{equation}
    G(s, \chi) = \sum_{p}\chi(p)p^{-s} + \sum_{p} \sum_{k=2}^{\infty} k^{-1}\chi(p^{k})p^{-ks}. \nonumber
\end{equation}
Since $\abs{\chi(p^{k})} \leq 1$, we may bound the second term in the same way as for $\log\zeta(s)$ by a function $R_{\chi}$ which remains bounded as $s \rightarrow 1$. Also note that $\chi(p) = 0$ when $p$ is a prime dividing $q$. Therefore
\begin{equation}
\label{GEquation1}
    G(s, \chi) = \sum_{p \nmid q}\chi(p)p^{-s} + R_{\chi}(s).
\end{equation}
Now, let $a$ be coprime to $q$, multiply both sides of (\ref{GEquation1}) by $\overline{\chi(a)}$, and sum over all Dirichlet characters modulo $q$. This gives 
\begin{equation}
\label{GEquation2}
    \sum_{\chi}\overline{\chi(a)}G(s, \chi) = \sum_{p \nmid q}p^{-s} \sum_{\chi}\chi(p)\overline{\chi(a)} + \sum_{\chi}\overline{\chi(a)}R_{\chi}.
\end{equation}
Now, we may apply the second orthogonality relation in Proposition~\ref{OrthogonalityRelations}: the sum over characters in the first term in the right hand side of (\ref{GEquation2}) gives $\phi(q)$ whenever $p \equiv a (q)$, and zero otherwise. Thus
\begin{equation}
\label{GEquation3}
    \sum_{\chi}\overline{\chi(a)}G(s, \chi) = \phi(q)\sum_{p \equiv a (q)}p^{-s} + R_{\chi, a}(s),
\end{equation}
where $R_{\chi, a}$ is again a function which is bounded as $s \rightarrow 1^{+}$. We are now very close to proving the theorem: the following proposition will allow us to complete the proof.
\begin{proposition}
\label{GDensity}
If $\chi_0$ denotes the trivial character, then
\begin{align}
    \lim_{s \rightarrow 1^{+}}\frac{G(s, \chi)}{\log(s - 1)^{-1}} = \left\{\begin{array}{lr}
        1 &  \textrm{if} \ \chi = \chi_{0}, \\
        0 &  \textrm{if} \ \chi \neq \chi_{0}. \\
        \end{array}\right\} \nonumber
\end{align}
\end{proposition}
The first case in this proposition is clear from (\ref{LZetaRelation}) - $\zeta(s)$ has a pole of identical residue at $s=1$ to $L(s, \chi_0)$, so this is equivalent to the Dirichlet density of all primes being 1. The latter statement is not so clear, and is equivalent to the fact that $L(1, \chi)$ is non-zero and analytic for non-trivial characters $\chi$ - let us assume this for now, as we will come back to this in more detail later. \\

If $L(1, \chi) \neq 0$ for non-trivial characters $\chi$, we may define $\log L(s, \chi)$ on a small interval $(1, 1 + \delta)$. Note that $G(s, \chi)$ and $\log L(s, \chi)$ are both mapped by $\exp$ to $L(s, \chi)$, so that $G(s, \chi) = \log L(s, \chi) + 2k\pi i $. Since $\log L(s, \chi)$ tends to a limit as $s \rightarrow 1^{+}$, $G(s, \chi)$ must remain bounded. Hence the second part of the proposition is proved. Thus, we divide both sides of (\ref{GEquation3}) by $\log (s - 1)^{-1}$ and take the limit as $s \rightarrow 1^{+}$. The limit on the left hand side is $1$ (the only term in the sum without limit zero is the one for the trivial character). On the right hand side, the first term approaches $\phi(q) d\left(\mathcal{P}(a; q)\right)$, while the second term goes to zero since the function $R_{\chi, a}$ remains bounded as $s \rightarrow 1^{+}$. We therefore divide both sides by $\phi(q)$, which gives Dirichlet's Theorem: 
\begin{equation}
    d\left(\mathcal{P}(a; q)\right) = 1/\phi(q). \nonumber
\end{equation}
We have therefore proved a deep theorem about primes using the properties of L-functions and their Dirichlet characters, assuming the key step that L-functions of non-trivial character are finite and non-zero at 1. The study of where the zeros of L-functions occur and do not occur is key to the study of the distribution of primes (and is in general hard!) - this theorem provides a taste of this. \\

\section{Introduction to the Prime Number Theorem}
In this section we refer to results in \cite{davenport, heath-brown_2005}. We turn our attention to the \textit{growth} of the prime numbers, and in particular the prime counting function, defined as
\begin{equation}
    \pi(x) \coloneqq \# \{p \leq x \ : \ p \ \textrm{prime}\}. \nonumber
\end{equation}
The aim of the Prime Number Theorem is to find the \textit{asymptotic distribution} of $\pi(x)$ - meaning a function which essentially approximates $\pi(x)$ well. Or, more precisely, one whose ratio with $\pi(x)$ tends to $1$ as $x$ becomes large. It turns out that it is more natural to study the function $\psi(x)$, defined as 
\begin{equation}
    \psi(x) \coloneqq \sum_{p^{k} \leq x} \log p = \sum_{n \leq x} \Lambda(n), \nonumber
\end{equation}
where $\Lambda(n)$ is the von Mangoldt function, defined as $\log p$ whenever $n$ is a power of a prime $p$, and zero otherwise. These two functions are related by the following lemma.
\begin{lemma}
For $x \geq 2$, 
\begin{equation}
\pi(x) = \frac{\psi(x)}{\log x} + \int_{2}^{x} \frac{\psi(t)}{t \log^{2} t} \mathrm{d} t + O(x^{1/2})
\end{equation}
\end{lemma}
\begin{proof}
First, set 
\begin{equation}
\theta(x) = \sum_{p \leq x} \log p. \nonumber
\end{equation}
Then we have
\begin{align}
\int_{2}^{x} \frac{\theta(t)}{t \log^{2} t} \mathrm{d} t &= \int_{2}^{x} \sum_{p \leq t} \frac{\log p}{t \log^{2} t} \mathrm{d} t \nonumber \\
&= \sum_{p \leq x} \int_{p}^{x}  \frac{\log p}{t \log^{2} t} \mathrm{d} t \nonumber \\
&= \sum_{p \leq x} \left[-\frac{\log p}{\log t} \right]_{p}^{x} \nonumber \\
&= \pi(x) - \frac{\theta(x)}{\log x}. \nonumber
\end{align}
Thus 
\begin{equation}
\label{piThetaRelation}
\pi(x) = \frac{\theta(x)}{\log x} + \int_{2}^{x} \frac{\theta(t)}{t \log^{2} t} \mathrm{d} t.
\end{equation}
Now consider the relationship between $\theta$ and $\psi$. We have that
\begin{equation}
\psi(x) = \theta(x) + \sum_{k \geq 2}\sum_{p^{k} \leq x} \log p. \nonumber
\end{equation}
In the double sum, the primes are at most $x^{1/2}$, so there are at most $x^{1/2}$ such terms, since the exponent is at least 2. Furthermore, having $p^{k} \leq x$ implies $k \leq \frac{\log x}{\log p}$. Therefore
\begin{align}
\psi(x) &\leq \theta(x) + \frac{\log x}{\log p} \sum_{p^{k} \leq x} \log p \leq \theta(x) + x^{1/2} \log x. \nonumber
\end{align}
So
\begin{equation}
\psi(x) = \theta(x) + O\left(x^{1/2} \log x \right). \nonumber
\end{equation}
Inserting this into (\ref{piThetaRelation}), we obtain the required result.
\end{proof}
For $a$ coprime to $q$, there is an analogous prime counting function for the residue class $a$ modulo $q$, defined as 
\begin{equation}
    \pi(x; q, a) \coloneqq \# \{p \leq x \ : \ p \equiv a \ (q), \ p \ \textrm{prime} \}, \nonumber
\end{equation}
as well as a corresponding $\psi$ function,
\begin{equation}
    \psi(x; q, a) \coloneqq \sum_{\substack{n \leq x \\ n \equiv a (q)}} \Lambda(n). \nonumber
\end{equation}
Lemma~{\ref{piThetaRelation}} also holds for $\pi(x; q, a)$ and $\psi(x; q, a)$, simply by restricting the sums to the relevant residue class. Remarkably, $\psi(x)$ and $\psi(x; q, a)$ are intimately connected with the Riemann zeta-function and Dirichlet L-functions respectively - in particular their zeros. Instead of studying these functions on a restricted region where they converge, we will use analytic continuation to study them on the whole of $\mathbb{C}$. In the case of $\psi(x)$, we shall see that it can be written explicitly for any $T \geq x \geq 1$ as
\begin{equation}
\label{ExpicitPsiFormula}
    \psi(x) = x - \sum_{\rho: \abs{\gamma} < T} \frac{x^{\rho}}{\rho} + O(\frac{x\log^{2}x}{T}),
\end{equation}
where the $\rho$ represent the (infinitely many) so-called non-trivial zeros of the analytically continued zeta-function. Standard notation for the $\rho$ is $\beta + i\gamma$, which we will use throughout. We shall see in the next section that these all lie in the  \textit{critical strip}, $0 \leq \sigma \leq 1$. The problem of estimating $\psi(x)$ and hence $\pi(x)$ is then essentially a question of estimating the contribution to this formula of the zeros as well as possible. A similar formula holds for the function
\begin{equation}
  \psi(x, \chi) \coloneqq \sum_{n \leq x}\chi(n)\Lambda(n),
\end{equation}
but this time over the non-trivial zeros of L-functions, which also lie in the critical strip. For $a$ coprime to $q$ we have
\begin{equation}
    \frac{1}{\phi(q)}\sum_{\chi} \overline{\chi}(a) \psi(x, \chi) = \frac{1}{\phi(q)}\sum_{n \leq x} \Lambda(n) \left(\sum_{\chi}\overline{\chi}(a) \chi(n) \right) = \sum_{\substack{n \leq x \\ n \equiv a (q)}} \Lambda(n) = \psi(x; q, a). \nonumber
\end{equation}
So, the analogous explicit formula for $\psi(x, \chi)$ reveals the relationship of the zeros of L-functions to $\psi(x; q, a)$, and hence the growth of primes in different residue classes. We will study the derivation and consequences of these formulae. \\

To proceed, we first study the analytic continuation of L-functions and the zeta-function to $\mathbb{C}$. Using this, we study both the frequency of these (analytically continued) functions' zeros and their position in the critical strip, which will lead to asymptotic distributions for the prime counting functions.
%NEW CHAPTER ANALYTIC CONTINUATION
\chapter{Analytic Continuation and the Functional Equations}
\section{Continuation to the Right Half-Plane}
Our general strategy in analytic continuation of L-functions, and in particular $\zeta(s)$, is to extend them to a slightly larger region first, before proving a powerful functional equation, which allows us to extend each function to $\mathbb{C}$.
First, consider $\zeta(s)$.
\begin{proposition}
For $\sigma > 0$,
\begin{equation}
\label{righthalfplanecontinuation}
\zeta(s) - \frac{1}{s-1} = \sum_{n=1}^{\infty} \left(\int_{n}^{n+1} (n^{-s} - x^{-s}) \mathrm{d}x \right),
\end{equation}
which gives the analytic continuation to the right half-plane.
\end{proposition}
\begin{proof}
We follow \cite{introduction-to-analytic-number-theory_2020}. The first part is to establish equality when $\sigma > 1$. First, note that 
\begin{align}
\sum_{n=1}^{k} \int_{n}^{n+1} x^{-s} \mathrm{d}x &= \frac{1}{1-s} \sum_{n=1}^{k} \left( (n+1)^{1-s} - n^{1-s} \right) \nonumber \\
&= \frac{1}{1-s} \left( -1^{1-s} + (2^{1-s} - 2^{1-s}) + \dots + (k^{1-s} - k^{1-s}) + (k+1)^{1-s} \right) \nonumber \\
&= \frac{1}{1-s} \left((k+1)^{1-s} - 1 \right) \rightarrow \frac{1}{s-1} \hspace{1mm} \textrm{as} \hspace{1mm} k \rightarrow \infty. \nonumber 
\end{align}
Therefore,
\begin{align}
\zeta(s) - \frac{1}{s-1} &= \sum_{n=1}^{\infty} n^{-s} - \sum_{n=1}^{\infty} \int_{n}^{n+1} x^{-s} \mathrm{d} x \nonumber \\
&= \sum_{n=1}^{\infty} \left( n^{-s} -  \int_{n}^{n+1} x^{-s} \mathrm{d} x \right) \nonumber \\
&= \sum_{n=1}^{\infty} \left(\int_{n}^{n+1} (n^{-s} - x^{-s}) \mathrm{d}x \right), \nonumber
\end{align}
since $\int_{n}^{n+1} n^{-s} \mathrm{d} x = n^{-s}$. Since equality has been established for $\sigma > 1$, we now must prove that the right hand side defines an analytic function on the larger region $\sigma > 0$. For $x \in [n, n+1]$,
\begin{equation}
\abs{n^{-s} - x^{-s}} = \abs{s\int_{n}^{x} y^{-1-s} \mathrm{d} y} = \abs{s}\abs{\int_{n}^{x} y^{-1-s} \mathrm{d} y}, \nonumber
\end{equation}
and we may estimate the integral as
\begin{align}
\abs{\int_{n}^{x} y^{-1-s} \mathrm{d} y} &\leq \abs{x-n} n^{-1-\sigma} \leq n^{-1-\sigma}, \nonumber
\end{align}
since $n \leq x \leq n+1$. Therefore,
\begin{equation}
\sum_{n=1}^{\infty} \abs{\int_{n}^{n+1} (n^{-s} - x^{-s}) \mathrm{d}x} \leq \sum_{n=1}^{\infty} \abs{s} n^{-1-\sigma} < \infty, \nonumber
\end{equation}
when $\sigma > 0$. Thus, the right hand side of (\ref{righthalfplanecontinuation}) defines an analytic function on the right half-plane, giving the analytic continuation of $\zeta(s)$ on this region.
\end{proof}
We remark that since we have written $\zeta(s)$ as $(s-1)^{-1}$ plus some analytic function on $\sigma > 0$, it follows that $\zeta(s)$ has residue $1$ at $s=1$, which was assumed in the previous chapter to prove facts about Dirichlet densities, and will be used repeatedly later. L-functions of trivial character have a similar extension to the right half plane by the relation (\ref{LZetaRelation}), so we continue to the case of L-functions of non-trivial character (which are to some extent easier to extend than $\zeta(s)$ owing to their lack of a singularity) and refer to \cite{ireland_rosen_1990}. First, define $S(x) = \sum_{n \leq x}\chi(n)$ for a non-trivial character $\chi$ to modulus $q$. For $\sigma > 1$, 
\begin{align}
    \sum_{n=1}^{N}S(n)\left(n^{-s} - (n + 1)^{-s}\right)
    &= L_{N}(s, \chi) - S(N)(N+1)^{-s} \rightarrow L(s, \chi) \ \textrm{as} \ N \rightarrow \infty \nonumber,
\end{align}
where $L_{N}(s, \chi)$ denotes the $N$-th partial sum of the series defining $L(s, \chi)$. Therefore:
\begin{align}
    L(s, \chi) &=  \sum_{n=1}^{\infty}S(n)\left(n^{-s} - (n + 1)^{-s}\right) \nonumber \\
    &= s \sum_{n=1}^{\infty} S(n) \int_{n}^{n+1} x^{-s-1} \mathrm{d} x \nonumber \\
    &= s \int_{1}^{\infty}S(x)x^{-s-1}\mathrm{d}x \quad (\sigma > 1). \nonumber
\end{align}
We now need the following Lemma.
\begin{lemma}
\label{CharacterSumBound}
For a non-trivial character $\chi$ to modulus $q$, define $S(x) = \sum_{n \leq x}\chi(n)$. Then $\abs{S(x)} \leq \phi(q)$ for all $x > 0$.
\end{lemma}
Assuming this Lemma gives
\begin{align}
    \abs{s \int_{1}^{\infty}S(x)^{-s-1}\mathrm{d}x} &\leq \abs{s}\int_{1}^{\infty}\abs{S(x)}x^{-\sigma - 1}\mathrm{d}x \nonumber \\
    &\leq \phi(q)\abs{s}\int_{1}^{\infty}x^{-\sigma - 1}\mathrm{d} x < \infty, \quad (\sigma > 0).
\end{align}
Therefore we have proved the following:
\begin{proposition}
For $\chi$ a non-trivial character to modulus $q$, and $S(x)$ defined as above, 
\begin{equation}
    L(s, \chi) = s \int_{1}^{\infty}S(x)x^{-s-1} \mathrm{d} x, \nonumber
\end{equation}
which defines an analytic function on $\sigma > 0$.
\end{proposition}
\begin{proof}
(Lemma~\ref{CharacterSumBound}) \\

For a given $x > 0$, write $x = m q + r$, where $0 \leq r < q$. By the $q$-periodicity of $\chi$,
\begin{equation}
    \sum_{n \leq x} \chi(n) = m\left(\sum_{n=0}^{q-1} \chi(n) \right) + \sum_{n \leq r} \chi(n). \nonumber
\end{equation}
By the first orthogonality relation in Proposition~\ref{OrthogonalityRelations}, the first term is 0. Hence
\begin{equation}
    \abs{S(x)} \leq \sum_{n \leq r} \abs{\chi(n)} \leq \phi(q), \nonumber
\end{equation}
where the last inequality comes from the fact that there are at most $\phi(q)$ non-zero terms, all of modulus 1.
\end{proof}
We have thus extended both $\zeta(s)$ and L-functions of non-trivial character to a slightly larger region. We wish to relate $\zeta(s)$ to $\zeta(1-s)$, and $L(s, \chi)$ to $L(1-s, \overline{\chi})$ respectively. This is done via so-called functional equations, and will complete the analytic continuation of these functions to the complex plane. 
\section{Gauss Sums and the Poisson Summation Formula}
The functional equation for L-functions involves a special type of function, known as the Gauss sum. Consider a character $\chi$ to modulus $q$. The \textit{Gauss sum} of $\chi$ is then defined as \begin{equation}
\label{GaussSum}
    \tau(\chi) \coloneqq \sum_{m=1}^{q} \chi(m) e^{2\pi i m/q}. 
\end{equation}
The following Lemma is an important property of Gauss sums. 
\begin{lemma}
For all $n \in \mathbb{N}$ and for all \textbf{primitive} characters $\chi$,
\begin{equation}
\label{GaussRelation}
    \chi(n) \tau(\overline{\chi}) = \sum_{h = 1}^{q} \overline{\chi}(h)e^{2\pi i n h / q} 
\end{equation}
\end{lemma}
\begin{proof}
For $n$ coprime to $q$, this property is clear:
\begin{align}
    \chi(n) \tau(\overline{\chi}) = \sum_{m=1}^{q} \overline{\chi}(m)\chi(n)e^{2 \pi i m / q}
    &= \sum_{m=1}^{q} \overline{\chi}(m n^{-1}) e^{2 \pi i m / q} = \sum_{h=1}^{q} \overline{\chi}(h) e^{2\pi i n h / q}, \nonumber 
\end{align}
where the last step sets $h = m n^{-1}$. If $n$ and $q$ are not coprime, the proof of this fact is rather more delicate, as the left hand side will be zero. It also requires $\chi$ to be primitive. However, proving that the right hand side is zero is not particularly enlightening, so the reader is referred to the appendix for details. 
\end{proof}
This Lemma has a useful Corollary. Multiplying both sides of (\ref{GaussRelation}) by their conjugates gives
\begin{align}
    \abs{\chi(n)}^{2}\abs{\tau(\overline{\chi})}^{2} &= \left(\sum_{h_1=1}^{q} \overline{\chi}(h_1) e^{2\pi i n h_1 / q} \right)\overline{\left(\sum_{h_2=1}^{q}\overline{\chi}(h_2) e^{2\pi i n h_2 / q} \right)} \nonumber \\
    &= \sum_{h_1=1}^{q}\sum_{h_2=1}^{q}\overline{\chi}(h_1)\chi(h_2) e^{2\pi i n (h_1 - h_2) / q}. \nonumber
\end{align}
Now sum over all residues modulo $q$. $\abs{\chi(n)}^{2}$ takes the value 1 at precisely $\phi(q)$ values, and is otherwise zero. Meanwhile, the sum of $e^{2\pi i n(h_1 - h_2)/q}$ over all $n$ is zero by symmetry, unless $h_1 \equiv h_2 \ (q)$, in which case it always takes the value 1. Thus, 
\begin{equation}
    \phi(q) \ \abs{\tau(\overline{\chi})}^{2} = q\sum_{h} \abs{\chi(h)}^{2} = q \ \phi(q), \nonumber
\end{equation}
which implies $\abs{\tau(\chi)} = q^{1/2}$ for all primitive characters $\chi$, upon swapping $\chi$ and $\overline{\chi}$. The importance of this result will be apparent later. \\

A second requirement in the study of the functional equation is the Poisson Summation Formula. We first define a class of functions on which the formula is valid. 
\begin{definition}
A smooth function is \textit{Schwartz} if for every $c \in \mathbb{R}$, $n \in \mathbb{N}\cup\{0\}$:
\begin{equation}
\abs{f^{(n)}(x)} = o(\abs{x}^{c}), \nonumber
\end{equation} 
where $f^{(n)}$ denotes the nth derivative of $f$. In other words, every derivative of the function decays quickly enough to dominate the growth rate of polynomials.
\end{definition}
We now state the theorem.
\begin{theorem}
(Poisson Summation Formula) Let $f : \mathbb{R} \rightarrow \mathbb{C}$ be a Schwartz function, and let $\hat{f}$ be its Fourier transform:
\begin{equation}
\hat{f}(y) = \int_{-\infty}^{\infty} e^{2 \pi i x y} f(x) \mathrm{d} y. \nonumber
\end{equation}
Then
\begin{equation}
\sum_{m \in \mathbb{Z}} f(m) = \sum_{n \in \mathbb{Z}} \hat{f}(n), \nonumber 
\end{equation}
with the sums converging absolutely.
\end{theorem}
\begin{proof}
Define $F(x) = \sum_{m \in \mathbb{Z}} f(x + m)$, and note that since $f$ is Schwartz, we have uniform convergence of the sum to a continuous function. To see this, we can bound each term by some arbitrarily large reciprocal power of $\abs{m}$ multiplied by some large constant, since $f(x+m) = o(\abs{x+m}^c)$ for all real numbers $c$. F is 1-periodic, so consider the Fourier coefficients of $F$:
\begin{align}
\int_{0}^{1} e^{-2 \pi i n x} F(x) \mathrm{d} x &= \int_{0}^{1} e^{-2 \pi i n x} \left( \sum_{m \in \mathbb{Z}} f(x + m) \right) \mathrm{d} x \nonumber \\
&= \sum_{m \in \mathbb{Z}} \int_{0}^{1} e^{-2 \pi i n x} f(x + m) \mathrm{d} x \hspace{1cm} \textrm{(By uniform convergence)} \nonumber \\
&= \sum_{m \in \mathbb{Z}} \int_{m}^{m+1} e^{-2\pi i n(y - m)}f(y) \mathrm{d} y \hspace{1cm} \textrm{(Substitute $x = y - m$)} \nonumber\\
&= \sum_{m \in \mathbb{Z}} \int_{m}^{m+1} e^{-2\pi i ny}f(y) \mathrm{d} y \hspace{1cm} \textrm{($e^{-2 \pi i n(y-m)} = e^{-2 \pi i ny}$)} \nonumber \\
&= \int_{-\infty}^{\infty} e^{-2 \pi i n y} f(y) \mathrm{d} y = \hat{f}(n).  \nonumber
\end{align}
So we have
\begin{equation}
F(x) = \sum_{n \in \mathbb{Z}} \hat{f}(n) e^{i n x}. \nonumber
\end{equation}
Now note that $F(0) = \sum_{m \in \mathbb{Z}} f(m)$ by definition. Therefore
\begin{equation}
\sum_{m \in \mathbb{Z}} f(m) = F(0) = \sum_{n \in \mathbb{Z}} \hat{f}(n). \nonumber
\end{equation}
\end{proof}
This theorem may be applied to $f(n) = e^{-(n + \alpha)^{2} \pi/x}$. Indeed, each derivative is dominated by an exponentially decaying function, so is clearly Schwartz. Thus, we find the Fourier transform: 
\begin{align}
    \hat{f}(n) &= \int_{-\infty}^{\infty} e^{2\pi i n t} f(t) \mathrm{d} t \nonumber \\
    &= \int_{-\infty}^{\infty} e^{2\pi i n t - (t + \alpha)^{2} \frac{\pi}{x}} \mathrm{d} t \nonumber \\
    &= e^{-n^{2} \pi x - 2 \pi i n \alpha} \int_{-\infty}^{\infty} e^{-\frac{\pi}{x}(t - i n x + \alpha)^2} \mathrm{d} t \nonumber \\
    &= \left(\frac{x}{\pi} \right)^{1/2} e^{-n^{2} \pi x - 2\pi i n \alpha}  \int_{-\infty}^{\infty} e^{-v^{2}} \mathrm{d} v \quad \left(\textrm{sub.} \ v = (x / \pi)^{1/2} (t - inx + \alpha) \ \right) \nonumber \\
    &= x^{1/2} e^{-n^{2} \pi x - 2\pi i n \alpha} \nonumber
\end{align}
By the Poisson summation formula, we therefore have the relation
\begin{equation}
\label{ModularRelation}
    \sum_{n=-\infty}^{\infty} e^{-(n + \alpha)^{2} \pi/x} = x^{1/2} \sum_{n=-\infty}^{\infty} e^{-n^{2} \pi x + 2\pi i n \alpha}.  
\end{equation}
We now have what we require to prove the functional equation for L-functions and the Riemann Zeta-function.

\section{The Functional Equations for $\zeta(s)$ and $L(s, \chi)$}
Throughout this section, we assume that $\chi$ is a primitive character of modulus $q$. Euler's Gamma function is defined for $\sigma > 0$ as 
\begin{equation}
    \Gamma(s) \coloneqq \int_{0}^{\infty}x^{s - 1} e^{-x} \mathrm{d} x. \nonumber
\end{equation}
The Gamma function is a generalisation of the factorial function to a continuous analytic function - one may prove by integration by parts that $\Gamma(s + 1) = s\Gamma(s)$ and $\Gamma(1) = 1$, so that $\Gamma(n) = (n - 1)! \ $ for positive integers $n$. Furthermore, repeated use of this relation allows us to give the analytic continuation of $\Gamma(s)$ to $\mathbb{C}$. Further properties include that the function is never zero, and has poles at the non-positive integers. Now, for $\sigma > 1$,
\begin{align}
    \label{FirstIdentity}
    \pi^{-s/2} q^{s/2} \Gamma(s/2) n^{-s} &= \left(\frac{q}{n^2 \pi}\right)^{s/2} \int_{0}^{\infty} y^{s/2 - 1} e^{-y} \mathrm{d} y \nonumber \\
    &= \int_{0}^{\infty} x^{s/2 - 1} e^{-n^{2} \pi x / q} \mathrm{d} x \quad (\textrm{set} \ y = n^{2}\pi x /q).
\end{align}
There are two main cases to consider depending on the value of $\chi(-1)$: note that since $\chi(-1)^{2} = \chi(1) = 1$, we have $\chi(-1) = \pm 1$. First, suppose $\chi(-1) = 1$. Multiplying both sides by $\chi(n)$ and summing over all positive integers $n$ gives
\begin{align}
    \pi^{-s/2} q^{s/2} \Gamma(s/2) L(s, \chi) &= \int_{0}^{\infty} x^{s/2 - 1} \left(\sum_{n=1}^{\infty}\chi(n) e^{-n^{2} \pi x / q} \right) \mathrm{d} x \nonumber \\
    &= \frac12 \int_{0}^{\infty} x^{s/2 - 1} \psi(x, \chi) \mathrm{d} x, \nonumber
\end{align}
where the interchange of summation and integration is justified by the absolute convergence of the left hand side on $\sigma > 1$, and the function $\psi$ is defined as
\begin{equation}
    \psi(x, \chi) \coloneqq \sum_{n=-\infty}^{\infty}\chi(n) e^{-n^{2}\pi x/q} = 2\sum_{n=1}^{\infty}\chi(n) e^{-n^{2}\pi x/q}. \nonumber
\end{equation}
Note that the last equality here relies on the fact that $\chi(n) = \chi(-1)\chi(n) = \chi(-n)$ and $\chi(0) = 0$. We now appeal to the properties of Gauss sums and the relation (\ref{ModularRelation}). Indeed,
\begin{align}
    \tau(\overline{\chi}) \psi(x, \chi) &= \sum_{n=-\infty}^{\infty} \tau(\overline{\chi}) \chi(n) e^{-n^2 \pi x / q} \nonumber \\
    &= \sum_{m=1}^{q} \overline{\chi}(m) \sum_{n=-\infty}^{\infty} e^{-n^{2} \pi x/q + 2\pi i m n / q} \quad (\textrm{by Lemma~\ref{GaussRelation}}). \nonumber
\end{align}
Replacing $x$ by $x/q$ and $\alpha$ by $m/q$ in (\ref{ModularRelation}), we therefore have
\begin{align}
\label{psiEquation1}
    \tau(\overline{\chi}) \psi(x, \chi) &= \sum_{m=1}^{q} \overline{\chi}(m) (q/x)^{1/2}\sum_{n=-\infty}^{\infty} e^{-(n + m/q)^{2}\pi q x^{-1}} \nonumber \\
    &= \left(\frac{q}{x}\right)^{1/2} \sum_{m=1}^{q}\overline{\chi}(m) \sum_{n=-\infty}^{\infty}e^{-(nq + m)^{2}\pi x^{-1}/ q} \nonumber \\
    &= \left(\frac{q}{x}\right)^{1/2} \sum_{k=-\infty}^{\infty} \overline{\chi}(k) e^{-k^{2} \pi x^{-1}/q} \nonumber \\
    &= \left(\frac{q}{x}\right)^{1/2} \psi(x^{-1}, \overline{\chi}).
\end{align}
Now, applying this relation gives
\begin{align}
\label{FirstIntegralEquation}
    \pi^{-s/2}q^{s/2}\Gamma(s/2)L(s, \chi) &= \int_{0}^{\infty} x^{s/2 - 1} \psi(x, \chi)\mathrm{d} x \nonumber \\
    &= \frac12 \int_{1}^{\infty} x^{s/2 - 1} \psi(x, \chi)\mathrm{d} x + \frac12 \int_{0}^{1} x^{s/2 - 1} \psi(x, \chi)\mathrm{d} x \nonumber \\
    &= \frac12 \int_{1}^{\infty} x^{s/2 - 1} \psi(x, \chi)\mathrm{d} x + \frac12 \int_{1}^{\infty} x^{-s/2 - 1} \psi(x^{-1}, \chi)\mathrm{d} x \nonumber \\
    &= \frac12 \int_{1}^{\infty} x^{s/2 - 1} \psi(x, \chi)\mathrm{d} x + \frac12\frac{q^{1/2}}{\tau(\overline{\chi})} \int_{1}^{\infty} x^{(1-s)/2 - 1} \psi(x, \overline{\chi})\mathrm{d} x.
\end{align}
Here, the penultimate step changes variables from $x$ to $x^{-1}$, while the last step appeals to (\ref{psiEquation1}). We now claim that the right hand side defines an analytic function on $\mathbb{C}$.  Indeed, 
\begin{align}
\abs{\int_{1}^{\infty} \psi(x, \chi) x^{s/2 - 1} \mathrm{d}x} &\ll \int_{1}^{\infty} \sum_{n=1}^{\infty} e^{-\pi n^{2} x} x^{\sigma/2 - 1} \mathrm{d}x \nonumber \\
& \ll \int_{1}^{\infty} \sum_{n=1}^{\infty} \frac{1}{n^2}  e^{-\frac{\pi x}{2}} x^{\sigma/2 - 1} \mathrm{d} x \nonumber \\
& \ll \int_{1}^{\infty} e^{-\frac{\pi x}{2}} x^{\sigma/2 - 1} \mathrm{d} x. \nonumber
\end{align}
There are two cases. If $\sigma \leq 2$, we have 
\begin{align}
\int_{1}^{\infty} e^{-\frac{\pi x}{2}} x^{\sigma/2 - 1} \mathrm{d} x \ll \int_{1}^{\infty} e^{-\frac{\pi x}{2}}\mathrm{d}x < \infty, \nonumber
\end{align}
and if $\sigma > 2$, 
\begin{align}
\int_{1}^{\infty} e^{-\frac{\pi x}{2}} x^{\sigma/2 - 1} \mathrm{d} x \ll \Gamma(\frac{\sigma}{2} - 1) < \infty \nonumber
\end{align}
by a change of variables. Thus each integral in (\ref{FirstIntegralEquation}) is analytic for all $s \in \mathbb{C}$, so their linear combination is also. Moreover, note that switching $s$ to $1 - s$, and $\chi$ to $\overline{\chi}$, we have
\begin{align}
  \frac12\frac{q^{1/2}}{\tau(\chi)} \int_{1}^{\infty} x^{s/2 - 1} \psi(x, \chi)\mathrm{d} x + \frac12 \int_{1}^{\infty} x^{(1-s)/2 - 1} \psi(x, \overline{\chi})\mathrm{d} x, \nonumber
\end{align}
which is the expression in (\ref{FirstIntegralEquation}) multiplied by $q^{1/2}/\tau(\chi)$. Indeed, since $\chi(n) = \chi(-n)$, we have
\begin{align}
    \overline{\tau(\chi)} &= \sum_{1}^{q} \overline{\chi}(m) e^{-2\pi i m/q} \nonumber \\
    &= \sum_{1}^{q} \overline{\chi}(m) e^{2\pi i m/q} \quad (\textrm{using} \ \chi(m) = \chi(-m)) \nonumber \\
    &= \tau(\overline{\chi}), \nonumber
\end{align}
so that $q = \abs{\tau(\chi)}^{2} = \tau(\chi) \tau(\overline{\chi})$, and the claim follows. Thus, we have obtained a functional equation in the case of a primitive character $\chi$ with $\chi(-1) = 1$. Defining 
\begin{equation}
    \xi_1(s, \chi) = \pi^{-s/2} q^{s/2} \Gamma(s/2) L(s, \chi), \nonumber
\end{equation}
therefore gives the relation
\begin{equation}
    \xi_1(1-s, \overline{\chi}) = \frac{q^{1/2}}{\tau(\chi)} \xi_{1}(s, \chi). \nonumber
\end{equation}
In the case where $\chi(-1) = -1$, we must proceed differently, as in this case the function $\psi(x, \chi)$ is zero! Shifting $s$ to $s + 1$, and multiplying by another factor of $n$ gives
\begin{align}
\label{xi2Integral}
    \xi_{2}(s, \chi) &\coloneqq \pi^{-(s + 1)/2}q^{(s + 1)/2} \Gamma\left(\frac12(s + 1)\right) L(s, \chi) \nonumber \\
    &= \frac12 \int_{0}^{\infty} \psi_{1}(x, \chi) x^{\frac{s}{2} - \frac12}\mathrm{d} x,
\end{align}
where we define 
\begin{equation}
    \psi_1(x, \chi) = \sum_{n=-\infty}^{\infty} n \chi(n) e^{-n^{2} \pi x / q}. \nonumber
\end{equation}
Using the same bounds in proving that $\xi_1(s, \chi)$ was analytic, both sides of (\ref{ModularRelation}) are uniformly convergent, allowing termwise differentiation with respect to $\alpha$. This gives
\begin{equation}
    \sum_{n=-\infty}^{\infty} n e^{-n^{2}\pi x + 2\pi i n \alpha} = i x^{-3/2} \sum_{n=-\infty}^{\infty}(n + \alpha) e^{-(n + \alpha)^{2}\pi/x}. \nonumber
\end{equation}
Setting $x$ to $x/q$ and $\alpha$ to $m/q$ as before, we obtain
\begin{align}
    \sum_{n=-\infty}^{\infty} n e^{\frac{-n^{2}\pi x}{q} + \frac{2\pi i n m}{q}} = i\left( \frac{q}{x} \right)^{3/2} \sum_{n=-\infty}^{\infty} (n + m/q) e^{-(n + m/q)^{2}\pi x^{-1} q}. \nonumber
\end{align}
Therefore following the exact same procedure as in the case of $\psi(x, \chi)$ gives
\begin{equation}
    \tau(\overline{\chi}) \psi_1(x, \chi) = i q^{1/2} x^{-3/2} \psi_{1}(x^{-1}, \overline{\chi}). \nonumber
\end{equation}
We then apply a similar splitting of the integral in (\ref{xi2Integral}) to obtain
\begin{align}
\label{SecondIntegralEquation}
    \xi_2(s, \chi) &= \frac12 \int_{1}^{\infty}\psi_1(x, \chi) x^{-(1 - s)/2} \mathrm{d} x + \frac12 \frac{i q^{1/2}}{\tau(\overline{\chi})}\int_{1}^{\infty} \psi_1(x, \overline{\chi})x^{-s/2} \mathrm{d} x.  
\end{align}
This is again analytic on $\mathbb{C}$. By an analogous argument to the previous case, we have $\overline{\tau(\chi)} = -\tau(\overline{\chi})$, so that $\tau(\chi)\tau(\overline{\chi}) = -q$. Thus, replacing $s$ by $1-s$ and $\chi$ by $\overline{\chi}$ in (\ref{SecondIntegralEquation}) yields
\begin{equation}
    \frac12 \frac{i q^{1/2}}{\tau(\chi)}\int_{1}^{\infty}\psi_1(s, \chi) x^{-(1 - s)/2} \mathrm{d} x + \frac12 \int_{1}^{\infty} \psi_1(x, \overline{\chi})x^{-s/2} \mathrm{d} x, \nonumber
\end{equation}
which is precisely (\ref{SecondIntegralEquation}) multiplied by $i q^{1/2}/\tau(\chi)$. We conclude that
\begin{equation}
    \xi_{2}(1-s, \overline{\chi}) = \frac{i q^{1/2}}{\tau(\chi)} \xi_{2}(s, \chi). \nonumber
\end{equation}
The two functional equations for $\xi_1$ and $\xi_2$ may be combined as follows. Define 
\begin{align}
a(\chi) = \left\{
    \begin{array}{ll}
         &  0 \ \textrm{if} \ \chi(-1) = 1\\
         &  1 \ \textrm{if} \ \chi(-1) = -1
    \end{array}
    \right\}.
    \nonumber
\end{align}
Then define 
\begin{equation}
    \xi(s, \chi) = (q/\pi)^{\frac12(s + a)} \Gamma\left(\frac12 (s + a) \right) L(s, \chi), \nonumber
\end{equation}
so that $\xi$ coincides with both $\xi_1$ and $\xi_2$, depending on the value of $a$. The two previous relations are then written compactly as
\begin{equation}
    \xi(1-s, \overline{\chi}) =  \frac{i^{a}q^{1/2}}{\tau(\chi)} \xi(s, \chi). \nonumber
\end{equation}
This completes the analytic continuation of L-functions of primitive characters to $\mathbb{C}$. Indeed, given a value $s$ on the left half plane, we have that $1-\sigma \geq 1$, on which region $L(1-s, \overline{\chi})$ is analytic. Since everything is therefore analytic, it is possible to rearrange to write $L(s, \chi)$ as a product of analytic functions on the left half plane. Note that dividing by the Gamma-function is valid as it is nowhere zero. \\

This argument is easily adapted to the case of $\zeta(s)$. Setting $q = 1$, we note that the trivial character is (trivially) primitive, so defining
\begin{equation}
    \xi(s) = \pi^{-s/2}\Gamma(s/2)\zeta(s), \nonumber
\end{equation}
we have the corresponding functional equation
\begin{equation}
    \xi(s) = \xi(1-s). \nonumber
\end{equation}
Again, since $\zeta(s)$ is analytic on the right half plane, this relation implies the analytic continuation of $\zeta(s)$ to $\mathbb{C}$. Now we have the continuations of $\zeta(s)$ and L-functions of primitive characters to the complex plane, we study their zeros. It is immediate from the functional equation that there are ``trivial zeros" of L-functions either at the negative even integers (and zero) or at the negative odd integers, depending on the value of $a$. These correspond to the poles of $\Gamma(s/2 + a/2)$, cancelling them to make an analytic function. \\

Furthermore, since L-functions are non-zero on $\sigma > 1$ (by the product formula), it follows that they are also non-zero on $\sigma < 0$ (if not coinciding with a trivial zero). Thus, any other zeros must lie in the \textit{critical strip} $0 \leq \sigma \leq 1$, and such zeros are deemed non-trivial. The exact same arguments hold for $\zeta(s)$. Having already mentioned the duality between the non-trivial zeros of L-functions and the primes, we now study both their frequency and positioning in the critical strip.

\chapter{The Zeros of $\zeta(s)$ and $L(s, \chi)$}
\section{Entire Functions of Finite Order}
Our strategy in this chapter follows that of \cite{davenport}: first, we aim to write $\xi(s)$ and $\xi(s, \chi)$ as infinite products in terms of their zeros. Then, we make conclusions with regard to the distribution of these zeros, and ultimately the distribution of primes. We begin by introducing a special class of function (the conditions for which the $\xi$ functions satisfy!).
\begin{definition}
An entire function $f(z)$ is of \textit{finite order} if it satisfies
\begin{equation}
f(z) = O( e^{\abs{z}^{\alpha} } ) \hspace{1mm} \textrm{as} \hspace{1mm} \abs{z} \rightarrow \infty
\end{equation} 
for some number $\alpha$. 
\end{definition}

Note that $\alpha > 0$ for all non-constant functions, so we can define the order of an entire function to be the infimum of such $\alpha$. We have the following elementary fact for entire functions of finite order which are nowhere zero.

\begin{proposition}
\label{no_zeros}
If $f(z)$ is an entire function of finite order, and $f(z) \neq 0$ for all $z \in \mathbb{C}$, then $f$ is necessarily of the form $e^{g(z)}$, where $g(z)$ is a polynomial. The order of $f$ is thus the degree of $g$.
\end{proposition}

\begin{proof}
We begin by defining $g(z) = \log f(z)$, and noting that $g(z)$ is also an entire function since $f$ is non-zero. On any large enough circle $\abs{z} = R$, we have the bound
\begin{equation}
\mathfrak{R}g(z) = \log \abs{f(z)} < 2R^{\alpha}, \nonumber
\end{equation}
where $\alpha$ is the order of the integral function $f$, using the property that $f(z) = O(e^{\abs{z}^{\alpha}})$. Away from the branch cut, $g(z)$ can be defined by a power series, written
\begin{equation}
g(z) = \sum_{n=0}^{\infty} (a_n + i b_n)z^n.  \nonumber
\end{equation}
Now for $z = R e^{i\theta}$, we have
\begin{equation}
\mathfrak{R}g(R e^{i\theta}) = \sum_{n=0}^{\infty} a_{n} R^{n} \cos(n\theta) - \sum_{n=1}^{\infty} b_{n} R^{n} \sin(n\theta). \nonumber
\end{equation}
This is a Fourier series for $\mathfrak{R}g(R e^{i\theta})$ as a function of $\theta$, so bounding its Fourier coefficients:
\begin{align}
\pi \abs{a_n} R^n &= \pi \abs{\frac{1}{2\pi} \int_{0}^{2\pi} \mathfrak{R}g(Re^{i\theta})\cos(n\theta) \mathrm{d} \theta }\nonumber \\ 
&\leq \int_{0}^{2\pi} \abs{\mathfrak{R} g(Re^{i\theta})} \mathrm{d} \theta \nonumber \\ 
&=  \int_{0}^{2\pi} \abs{\mathfrak{R} g(Re^{i\theta})} + \mathfrak{R} g(Re^{i\theta}) \mathrm{d} \theta - 2\pi a_0\nonumber \\ 
&\leq C\pi R^{\alpha}, \nonumber
\end{align}
where $C$ is some absolute constant independent of $R$. Therefore
\begin{equation}
\abs{a_n} < CR^{\alpha - n}, \nonumber
\end{equation}
so we can let $R \rightarrow \infty$ to conclude $a_{n} = 0$ for $n > \alpha$. We can bound the $b_{n}$ similarly, so we conclude that $g(z)$ is a polynomial of degree less than or equal to $\alpha$. Hence the order of $f$ is simply the degree of $g$.
\end{proof}

We now consider the case where an entire function $f(z)$ has zeros at $z_1, z_2, \dots $ and we wish to relate the distribution of its zeros to its order, say $\rho$. It will be important when writing $f$ as an infinite product that its zeros are sufficiently far apart. The easiest way to answer the question of the distribution of its zeros is by a formula due to Jensen. 
\begin{proposition}
\label{Jensen}
(Jensen's Formula) If $f$ is an entire function with $f(0) \neq 0$ and zeros at $z_1, z_2, \dots$, none of which are on $\abs{z} = R$, then
\begin{equation}
\frac{1}{2\pi} \int_{0}^{2\pi} \log\abs{f(Re^{i\theta})}\mathrm{d}\theta = \int_{0}^{R} r^{-1} n(r) \mathrm{d} r, \nonumber
\end{equation}
where $n(r)$ is the number of zeros of $f$ with radius less than $r$. 
\end{proposition}
\begin{proof}
We flesh out the proof in \cite{ivic_2003}. Firstly, using the assumption that $f(0) \neq 0$,
\begin{align}
\frac{1}{2\pi} \int_{0}^{2\pi} \log\abs{f(Re^{i\theta})}\mathrm{d}\theta - \log \abs{f(0)} &= \frac{1}{2\pi} \int_{0}^{2\pi} \mathfrak{R}\log f(Re^{i\theta})\mathrm{d}\theta - \log \abs{f(0)} \nonumber \\
&= \frac{1}{2\pi} \int_{0}^{2\pi} \left(\mathfrak{R}\int_{0}^{R}\frac{\mathrm{d}}{\mathrm{d}r}\log f(re^{i\theta}) \mathrm{d} r \right)\mathrm{d}\theta \nonumber \\
&= \frac{1}{2\pi}\mathfrak{R} \int_{0}^{2\pi} \int_{0}^{R}\frac{f'(re^{i\theta}) e^{i\theta}}{f(re^{i\theta})} \mathrm{d} r \mathrm{d}\theta \nonumber \\
&= \mathfrak{R}\int_{0}^{R}\frac{1}{2\pi i r}  \int_{0}^{2\pi}\frac{f'(re^{i\theta}) i r e^{i\theta}}{f(re^{i\theta})} \mathrm{d}\theta \mathrm{d}r  \nonumber \\
&= \mathfrak{R} \int_{0}^{R} r^{-1} \left( \frac{1}{2\pi i} \oint_{\abs{z}=r} \frac{f'(z)}{f(z)} \mathrm{d} z \right)\mathrm{d}r. \nonumber
\end{align}
Now, by the argument principle and the fact that $f$ is entire so has no poles,
\begin{equation}
\frac{1}{2\pi i} \oint_{\abs{z}=r} \frac{f'(z)}{f(z)} \mathrm{d} z  = n(r), \nonumber
\end{equation}
where $n(r)$ represents the number of zeros of $f$ inside the ball of radius $r$ centred at zero. We therefore conclude that
\begin{equation}
\frac{1}{2\pi} \int_{0}^{2\pi} \log\abs{f(Re^{i\theta})}\mathrm{d}\theta - \log \abs{f(0)}= \int_{0}^{R} r^{-1} n(r) \mathrm{d}r \nonumber
\end{equation}
\end{proof}
We now put this formula to immediate use. Suppose as before that $f$ is an entire function of order $\rho$ with $f(0) \neq 0$. By the bound in the proof of Proposition~\ref{no_zeros}, we have $\log\abs{ f(Re^{i\theta})} < R^{\alpha}$ on some large enough $R$ for any $\alpha > \rho$. Invoking Jensen's formula gives
\begin{align}
\int_{0}^{R} r^{-1} n(r) \mathrm{d}r \ll \frac{1}{2\pi} \int_{0}^{2\pi} \log\abs{f(Re^{i\theta})}\mathrm{d}\theta
&< 2R^{\alpha}. \nonumber
\end{align}
Since $n(r)$ is an increasing function of $r$, we have
\begin{equation}
\int_{R}^{2R} r^{-1} n(r) \mathrm{d} r \geq n(R) \int_{R}^{2R} r^{-1} \mathrm{d} r = n(R) \log 2. \nonumber
\end{equation}
From this, it follows that 
\begin{align}
n(R) \leq \frac{1}{\log 2} \int_{R}^{2R} r^{-1} n(r) \mathrm{d} r &\leq \frac{1}{\log 2} \int_{0}^{2R} r^{-1} n(r) \mathrm{d} r \ll R^{\alpha}, \nonumber
\end{align}
so we conclude that
\begin{equation}
\label{zeros_bound}
n(R) = O(R^{\alpha})
\end{equation}
for any $\alpha > \rho$. This tells us that an entire function $f(z)$ must have its zeros sufficiently far apart. Consequently, if each zero $z_n$ has radius $r_n$, we know that for any $\beta > \alpha$, $\sum_{n=1}^{\infty} r_{n}^{-\beta}$ converges. To see this, we split the sum into different annuli:
\begin{align}
\sum_{n=1}^{\infty} r_{n}^{-\beta} &= \sum_{k=0}^{\infty} \left(\sum_{2^{k} \leq r_{n} < 2^{k + 1}} r_{n}^{-\beta}\right) \leq \sum_{k=0}^{\infty} \left(\sum_{2^{k} \leq r_{n} < 2^{k + 1}} 2^{-\beta k}\right) \nonumber
\end{align}
The number of terms in each of the sums is $O(2^{\alpha k })$ by (\ref{zeros_bound}). Thus
\begin{equation}
\sum_{n=1}^{\infty} r_{n}^{-\beta} \ll \sum_{k=0}^{\infty} 2^{k(\alpha - \beta)} < \infty, \nonumber
\end{equation}
since we assumed $\beta > \alpha$. This therefore holds for any $\beta > \rho$, since $\alpha$ can be made arbitrarily close to $\rho$. 
\section{The Hadamard Product Formula}
We will now be chiefly concerned with entire functions $f(z)$ of order $\rho = 1$, and zeros at points $z_1, z_2, \dots$. By the results of the previous section, we know that $\sum_{n=1}^{\infty} r_{n}^{-1-\varepsilon}$ converges for any $\varepsilon > 0$, so in particular $\sum_{n=1}^{\infty} r_{n}^{-2}$ converges. We prove the following:
\begin{theorem}
\label{HadamardTheorem}
(The Hadamard Product Formula) For an entire function $f(z)$ of order 1, with zeros at $z_1, z_2, \dots$, there exist constants $A, B$ such that
\begin{equation}
    f(z) = e^{A + B z} \prod_{n=1}^{\infty} (1 - z/z_n) e^{z/z_n}. \nonumber 
\end{equation}
\end{theorem}
Before proceeding, we must first check that the infinite product term has nice properties. Firstly, the infinite product 
\begin{equation}
P(z) = \prod_{n=1}^{\infty} (1-z/z_n)e^{z/z_n}
\end{equation}
is absolutely convergent, and is uniformly convergent in compact sets containing no $z_n$. To show this, note that if $\log P(z)$ is convergent, then so is $P(z)$. We have
\begin{equation}
\log P(z) = \sum_{n=1}^{\infty} \log (1 - z/z_n) + z/z_n \nonumber, 
\end{equation}
so by the Taylor expansion of $\log(1-z)$, we have 
\begin{align}
\log P(z) &= \sum_{n=1}^{\infty} z/z_n + \left( -(z/z_n) - \frac{(z/z_n)^{2}}{2} - \dots \right) \nonumber \\
&=  \sum_{n=1}^{\infty} \sum_{k=2}^{\infty} \frac{-(z/z_n)^{k}}{k} \nonumber \\
&= z^{-2}\sum_{n=1}^{\infty} z_n^{-2} \sum_{k=0}^{\infty} \frac{-(z/z_n)^{k}}{k+2}.
\end{align}
The absolute value of the second infinite sum is always bounded by $\log\abs{1-z/z_n}$, so we have 
\begin{align}
\abs{\log P(z)} \leq \abs{z}^{-2} \sum_{n=1}^{\infty} r_n^{-2} \log\abs{1-z/z_n} < \infty, \nonumber
\end{align}
whenever $z$ is not equal to any of the $z_n$. Therefore $\log P(z)$ is absolutely convergent which implies $P(z)$ is convergent as required. We have therefore found an entire function $P(z)$ with exactly the same zeros (with multiplicity) as $f(z)$. Now, write
\begin{equation}
\label{f_product}
    f(z) = F(z)P(z).
\end{equation}
We have that $F(z)$ is also an entire function, and crucially one without zeros. In order to invoke Proposition~\ref{no_zeros}, we must prove that $F(z)$ is finite order. We proceed by finding a lower bound for $\abs{P(z)}$, and hence an upper bound for $\abs{F(z)}$ on a sequence of circles $\abs{z}=R$. \\

We must keep the values of $R$ away from the zeros $r_n$. Since $\sum r_n^{-2}$ converges, the total length of all the intervals $(r_n - r_n^{-2}, \hspace{1mm} r_n + r_n^{-2})$ is finite. Hence it cannot cover any infinite part of the real line. In particular, for each $r > 0$, there is an $R > r$ such that $R$ does not lie in one of the intervals. In other words, there are arbitrarily large values of $R$ such that 
\begin{equation}
    \abs{R - r_n} > r_n^{-2}. \nonumber
\end{equation}
for every $n \in \mathbb{N}$. We proceed by fixing a large value $R$, and splitting the infinite product $P(z)$ into subproducts, depending on the radius of each zero. Define
\begin{equation}
    P(z) = P_{1}(z)P_{2}(z)P_{3}(z), \nonumber
\end{equation}
where each $P_{i}$ is the product over the following sets of $z_n$:
\begin{align}
    P_1(z): & \hspace{5mm} \abs{z_n} < \frac{1}{2}R, \nonumber \\
    P_2(z): & \hspace{5mm} \frac{1}{2}R \leq \abs{z_n} \leq 2R, \nonumber \\
    P_3(z): & \hspace{5mm} \abs{z_n} > 2R. \nonumber
\end{align}
For the terms of $P_1(z)$, on $\abs{z} = R$ we have 
\begin{align}
    \abs{(1-z/z_n)e^{z/z_n}} &\geq (\abs{z/z_n} - 1) e^{\mathfrak{R}(z/z_n)}
    \geq (\abs{z/z_n} - 1) e^{-\abs{z/z_n}}
    > e^{-R/r_n} \nonumber,
\end{align}
using the fact that $\abs{z/z_n} > 2$, so that the multiplying factor of the exponential term is greater than 1. We also have that
\begin{align}
    \sum_{r_n < R/2} r_n^{-1} &= \sum_{r_n < R/2} r_{n}^{-1-\varepsilon} \ r_n^{\varepsilon} < \left( \frac12 R \right)^{\varepsilon} \sum_{r_n < R/2} r_n^{-1-\varepsilon}
    = C \left( \frac12 R \right)^{\varepsilon}, \nonumber 
\end{align}
for some constant $C$, since the (possibly infinite) sum will always converge for any $\varepsilon > 0$. Therefore, 
\begin{align}
    \abs{P_1(z)} = \prod_{r_n < R/2} \abs{(1 - z/z_n) e^{z/z_n}} 
    > \exp{\left(-R\left( \sum_{r_n < R/2} r_n^{-1} \right)\right)}
    > \exp{(-R^{1 + 2\varepsilon})}, \nonumber
\end{align}
for $\varepsilon$ small enough. Now, for the terms in $P_2(z)$, we have
\begin{align}
    \abs{(1-z/z_n)e^{z/z_n}} &= \frac{\abs{z_n - z}}{\abs{z_n}} e^{\mathfrak{R}(z/z_n)} \geq \frac{e^{-2} \abs{z_n - z}}{2R} > CR^{-3}, \nonumber
\end{align}
for some constant $C>0$, where the last inequality uses the fact that we chose our $z$ to be greater than $r_n^{-2}$ in distance away from every $z_n$. Therefore the least distance between any $z_n$ and $z$ is proportional to $\abs{z}^{-2} = R^{-2}$. Using (\ref{zeros_bound}), we have that the number of zeros between $R/2$ and $2R$, and hence the number of terms in the product $P_2(z)$, is $O(R^{1 + \varepsilon})$. Therefore
\begin{align}
    \abs{P_2(z)} > \exp{\left( -R^{1 + \varepsilon}(\log CR^3) \right)} > \exp{\left( -R^{1 + 2\varepsilon} \right)}, \nonumber
\end{align}
for $R$ large enough. Finally, for the terms of $P_3(z)$, we have 
\begin{align}
    \abs{(1 - z/z_n) e^{z/z_n}} &\geq \abs{1 - \abs{z/z_n}} e^{\mathfrak{R}(z/z_n)} > \frac12 e^{\mathfrak{R}(z/z_n)} > e^{-c\left( R/r_n \right)^{2}}, \nonumber
\end{align}
for some $c > 0$ which absorbs the factor of $\frac12$, and the extra factor of $R/r_n$ in the exponent, which is bounded above by $\frac12$. We also have
\begin{align}
    \sum_{r_n > 2R} r_n^{-2} = \sum_{r_n > 2R} r_n^{-1-\varepsilon} r_n^{-1+\varepsilon} &< \left( 2R \right)^{-1+\varepsilon} \sum_{r_n > 2R} r_n^{-1-\varepsilon} = C(2R)^{-1+\varepsilon}, \nonumber
\end{align}
for some constant $C>0$, which finally gives
\begin{align}
    \abs{P_3(z)} &> \prod_{r_n > 2R} e^{-c\left( R/r_n \right)^{2}} \nonumber \\
    &= \exp{\left( -c R^{2} \sum_{r_n > 2R} r_n^{-2} \right)} \nonumber \\
    &> \exp{\left( -C' R^{2} (2R)^{-1 + \varepsilon} \right)} \nonumber \\
    &> \exp{\left( -R^{1 + 2\varepsilon} \right)}, \nonumber
\end{align}
for $R$ sufficiently large. Therefore we can conclude that 
\begin{align}
    \abs{P(z)} &= \abs{P_1(z)}\abs{P_2(z)}\abs{P_3(z)} > \exp{(-3R^{1 + 2\varepsilon})} > \exp{(-R^{1 + 3\varepsilon})}, \nonumber
\end{align}
again for $R$ large enough. Since $f(z)$ was order 1 by assumption, we have $f(z) < \exp{(R^{1 + \varepsilon})}$ for each $\varepsilon > 0$, $R$ sufficiently large. Then we have that
\begin{equation}
    \abs{F(z)} = \abs{f(z)}\abs{P(z)}^{-1} < \exp{(R^{1 + 4\varepsilon})}. \nonumber
\end{equation}
Since $\varepsilon$ can be made arbitrarily small, we conclude that $F(z)$ is in fact entire of order 1 with no zeros. Invoking Proposition~\ref{no_zeros}, we have that $F(z) = e^{A + B z}$ for some constants $A, B$, leading us to Hadamard's formula,
\begin{equation}
\label{hadamard}
    f(z) = e^{A + B z} \prod_{n=1}^{\infty} (1 - z/z_n) e^{z/z_n}.
\end{equation}
\section{Infinite Product Representation of $\xi(s)$ and $\xi(s, \chi)$}

The formula (\ref{hadamard}) turns out to be very useful indeed. It allows us to write the $\xi$ functions, and hence L-functions, in terms of their zeros. This is key to studying their distribution, and hence vital in proof of the prime number theorem. Owing to the pole of $\zeta(s)$ at $s=1$, we must treat $\xi(s)$ and $\xi(s, \chi)$ separately. It is convenient in this section to define $\xi(s)$ as
\begin{equation}
    \xi(s) = \frac12 s(s-1) \pi^{-s/2} \Gamma(s/2) \zeta(s), \nonumber
\end{equation}
so that $\xi(s)$ is entire. Note that the relation $\xi(s) = \xi(1-s)$ holds as before. Therefore,
\begin{equation}
    \xi(0) = \xi(1) = \frac12 \pi^{-1/2} \Gamma(1/2) \{ \textrm{Res}_{s=1} \zeta(s) \} = \frac12. \nonumber
\end{equation}
Furthermore, the trivial zeros of $\zeta(s)$ are cancelled by the poles of $\Gamma(s/2)$, so the only zeros of $\xi(s)$ are the non-trivial zeros of $\zeta(s)$. Now, when $\frac12 < \sigma < 2$, it is clear that $\xi(s)$ is bounded, since the only pole of any of its components is of $\zeta(s)$ at $s=1$, at which point $\xi(s)$ is finite anyway. When $\sigma \geq 2$, by definition we have $\zeta(s) = O(1)$, while Stirling's approximation (see appendix) gives $\Gamma(s/2) = O(\exp{\abs{s}\log\abs{s}})$. Therefore
\begin{equation}
    \xi(s) = O(\exp{(\abs{s}\log\abs{s})}). \nonumber
\end{equation}
This implies that $\xi(s)$ is of order at most 1, with zeros at the non-trivial zeros of the Zeta-function. Applying (\ref{hadamard}), we have
\begin{equation}
    \xi(s) = e^{A + B z} \prod_{\rho}(1 - s/\rho) e^{s/\rho}, \nonumber
\end{equation}
where the $\rho$ are the non-trivial zeros of $\zeta(s)$, and $A, B$ are constants. Now, we can take appropriate branches of logarithm which yields
\begin{equation}
    \log \xi(s) = A + B z + \sum_{\rho} \left( \log(1 - s/\rho) + s/\rho \right). \nonumber
\end{equation}
We already proved in generality that the logarithm of the infinite product representation is uniformly convergent for integral functions of order 1 in compact sets containing no zeros $\rho$, so we can differentiate termwise to give 
\begin{equation}
    \frac{\xi'(s)}{\xi(s)} = B + \sum_{\rho} \left( \frac{1}{s-\rho} + \frac{1}{\rho} \right). \nonumber
\end{equation}
On the other hand, logarithmic differentiation from the definition of $\xi(s)$ gives
\begin{align}
    \frac{\xi'(s)}{\xi(s)} &= \frac{\mathrm{d}}{\mathrm{d}s} \left( - \frac{s}{2}\log \pi + \log (s-1) + \log \Gamma(s/2 + 1) + \log \zeta(s) \right) \nonumber \\
    &= -\frac12 \log \pi + \frac{1}{s-1} + \frac12 \frac{\Gamma'(s/2 + 1)}{\Gamma(s/2 + 1)} + \frac{\zeta'(s)}{\zeta(s)}, \nonumber
\end{align}
where we absorb the factor of $s/2$ into the $\Gamma$-function. This implies that
\begin{equation}
\label{ZetaPartialFraction}
    \frac{\zeta'(s)}{\zeta(s)} = -\frac{1}{s-1} + B + \frac12 \log \pi - \frac12 \frac{\Gamma'(s/2 + 1)}{\Gamma(s/2 + 1)} + \sum_{\rho} \left( \frac{1}{s-\rho} + \frac{1}{\rho} \right).
\end{equation}
In the case of $\xi(s, \chi)$ for a primitive character $\chi$ modulo $q$, the function is already entire, so there is no need to cancel any poles. Thus, there is an analogous formula for L-functions derived in exactly the same way, namely
\begin{equation}
\label{LPartialFraction}
    \frac{L'(s, \chi)}{L(s, \chi)} = -\frac12 \log \frac{q}{\pi} + B(\chi) - \frac12 \frac{\Gamma'(s/2 + a/2)}{\Gamma(s/2 + a/2)} + \sum_{\rho} \left( \frac{1}{s-\rho} + \frac{1}{\rho} \right), 
\end{equation}
where $\rho$ are the non-trivial zeros of $L(s, \chi)$. It is through these formulae where we begin to see the relevance of their zeros to the primes: we may write L-functions as infinite products over the primes, which consequently through logarithmic differentiation yields an expression directly involving their non-trivial zeros.
\section{A Zero-Free Region for $\zeta(s)$}
Recall the formula (\ref{ExpicitPsiFormula}). Estimating $\psi(x)$ well, and hence $\pi(x)$, essentially boils down to two things. The first is making the contribution of each individual non-trivial zero, i.e. $\abs{x^{\rho}}$, as small as possible. To ensure this, we aim to find minimum distance $\rho$ must lie from the line $\sigma = 1$ in the critical strip, so that each term is lower order than $x$. The second way of ensuring the contribution of the zeros is minimised is to quantify how many terms there are in the sum. In this section, we will be concerned with the former. Such a minimum distance of each zero from the 1-line is known as a \textit{zero-free region}. The case of trivial and non-trivial characters must be treated separately, owing to the pole at $s=1$ for the trivial character. \\

In any case, our work so far has dealt mainly with primitive characters, so for the trivial character we refer to $\zeta(s)$. Recall (\ref{logZeta}), where
\begin{equation}
    \log\zeta(s) = \sum_{p}\sum_{m=1}^{\infty}m^{-1}p^{-m s} \quad (\sigma > 1). \nonumber
\end{equation}
Since the right hand side is uniformly convergent on $\sigma > 1$, differentiating termwise gives
\begin{equation}
    \frac{\zeta'(s)}{\zeta(s)} = -\sum_{p}\log p \left(\sum_{m=1}^{\infty}p^{-ms} \right) = -\sum_{n=1}^{\infty}\Lambda(n) n^{-s}, \quad (\sigma > 1)\nonumber
\end{equation}
where $\Lambda(n)$ is the von Mangoldt function defined in chapter 2. Therefore
\begin{equation}
\label{RealZetaOverZeta}
    -\mathfrak{R}\frac{\zeta'(s)}{\zeta(s)} =  \sum_{n=1}^{\infty}\Lambda(n) n^{-\sigma} \cos(t \log n) \quad (\sigma > 1).
\end{equation}
Now, consider the inequality, valid for all $\theta$,
\begin{equation}
    0 \leq 2(1 + \cos\theta)^{2} = 3 + 4\cos\theta + \cos 2\theta. \nonumber
\end{equation}
Applied to (\ref{RealZetaOverZeta}), it implies that for all $t$,
\begin{equation}
\label{LinCombZetaInequality}
    3\left(-\frac{\zeta'(\sigma)}{\zeta(\sigma)}\right) + 4\left(-\mathfrak{R}\frac{\zeta'(\sigma + i t)}{\zeta(\sigma + i t)}\right) + \left(-\mathfrak{R}\frac{\zeta'(\sigma + 2i t)}{\zeta(\sigma + 2i t)}\right) \geq 0.
\end{equation}
We now wish to bound each term by above in terms of the non-trivial zeros. Firstly, since $\zeta(s)$ has a simple pole at $s=1$ of residue 1, it follows that for $1 < \sigma \leq 2$:
\begin{equation}
    -\frac{\zeta'(\sigma)}{\zeta(\sigma)} < \frac{1}{\sigma - 1} + A_1, \nonumber
\end{equation}
for some positive constant $A_1$. For complex $s$ with $1 < \sigma \leq 2$, $t \geq 2$, we refer back to the equality (\ref{ZetaPartialFraction}). Since Stirling's formula implies $\log\Gamma(s) \sim s\log s$, we have
\begin{equation}
    \abs{\frac12\frac{\Gamma'(s/2 + 1)}{\Gamma(s/2 + 1)}} < A_2\log t, \nonumber
\end{equation}
for a positive constant $A_2$. Since all other terms in (\ref{ZetaPartialFraction}) except the sum are bounded,
\begin{equation}
\label{Chapter4Inequality1}
    -\mathfrak{R}\frac{\zeta'(s)}{\zeta(s)} < A_2\log t - \sum_{\rho} \mathfrak{R}\left( \frac{1}{s-\rho} + \frac{1}{\rho} \right).
\end{equation}
Furthermore, since
\begin{equation}
    \mathfrak{R}\frac{1}{s - \rho} = \frac{\sigma - \beta}{\abs{s-\rho}^{2}}, \quad \textrm{and} \quad \mathfrak{R}\frac{1}{\rho} = \frac{\beta}{\abs{\rho}^{2}}, \nonumber
\end{equation}
the sum over zeros is strictly positive. We can therefore throw away any number of terms in (\ref{Chapter4Inequality1}), and the inequality will still hold. Choose $t$ such that $t$ coincides with the imaginary part of a non-trivial zero, say $\rho_1 = \beta + i\gamma$. Then we certainly have
\begin{equation}
    -\mathfrak{R}\frac{\zeta'(\sigma + 2 i t)}{\zeta(\sigma + 2 i t)} < A_{2}\log t, \nonumber
\end{equation}
and 
\begin{equation}
     -\mathfrak{R}\frac{\zeta'(\sigma + i t)}{\zeta(\sigma + i t)} < A_{2}\log t - \frac{1}{\sigma - \beta},
\end{equation}
where the second inequality throws away all but the $(s-\rho)^{-1}$ term in the sum corresponding to the zero $\rho_1$. Putting all of these inequalities into (\ref{LinCombZetaInequality}), we obtain
\begin{equation}
    \frac{4}{\sigma - \beta} < \frac{3}{\sigma - 1} + A \log t, \nonumber
\end{equation}
where $A$ is a positive constant. Let $\sigma = 1 + \delta / \log t$. Then, sparing tedious algebraic details, we have
\begin{equation}
   \beta < 1 - \frac{\delta(1 - A\delta)}{(3 + A \delta) \log t}, \nonumber
\end{equation}
which, upon choosing $\delta$ small enough in relation to $A$, gives an absolute constant $c$ such that
\begin{equation}
    \beta < 1 - \frac{c}{\log t}. \nonumber
\end{equation}
Since $\beta + i \gamma$ was an arbitrary zero, we may therefore find a $c$ for any $t \geq 2$ such that any zero satisfies this property. It follows from the functional equation that if $\rho$ is a non-trivial zero of $\zeta(s)$, then so is $\overline{\rho}$. Therefore the above region holds upon replacing $t$ by $\abs{t}$. Hence, we have found an explicit zero free region for $\zeta(s)$, and proved the following.
\begin{theorem}
For all $s=\sigma + it$ satisfying
\begin{equation}
    \sigma \geq 1 - \frac{c}{\log \abs{t}}, \quad \abs{t} \geq 2, \nonumber
\end{equation}
we have $\zeta(s) \neq 0$. 
\end{theorem}
In order to extend this theorem to all $t$, we must prove that there are no zeros arbitrarily close to $\sigma = 1$ where $\abs{t} < 2$. This is implied by the statement that $\zeta(1 + i t) \neq 0$ for all $t$. To show this, note that an identical argument as before involving the (double) infinite sum representation of $\log \zeta(s)$ in (\ref{logZeta}) gives
\begin{equation}
    3\log\zeta(\sigma) + 4\mathfrak{R}\log\zeta(\sigma + it) + \mathfrak{R} \log \zeta(\sigma + 2it) \geq 0, \nonumber
\end{equation}
so that exponentiation implies
\begin{equation}
\label{ZetaPowerRelation}
    \abs{\zeta^{3}(\sigma)\zeta^{4}(\sigma + it)\zeta(\sigma + 2it)} \geq 1,
\end{equation}
all of this when $\sigma > 1$. Suppose $\zeta(1 + it) = 0$. As $\sigma \rightarrow 1$, $\zeta(\sigma) \sim (\sigma - 1)^{-1}$, while $\zeta(\sigma + it) \sim (\sigma - 1)$ (by assumption), so that the left hand side of  (\ref{ZetaPowerRelation}) goes to zero. This clearly contradicts (\ref{ZetaPowerRelation}), so we conclude that $\zeta(1 + it)$ is non-zero. Thus, we may extend the zero free region to all $s$ satisfying
\begin{equation}
    \sigma > 1 - \frac{c}{\log(\abs{t} + 2)}. \nonumber
\end{equation}
\section{A Zero-Free Region for \texorpdfstring{$L(s, \chi)$}{Lg}}
We now consider L-functions of a non-trivial character, since by (\ref{LZetaRelation}) a zero-free region for those of trivial character follows immediately from that of $\zeta(s)$. An added complication in this case is the value of $q$, as such regions will depend on the size of this modulus. Suppose $\chi$ is a non-trivial character of modulus $q \geq 3$, and that $t \geq 0$. It is enough to consider non-negative $t$, since any zero of $L(s, \chi)$ with $t < 0$ is a zero of $L(s, \overline{\chi})$ with $t > 0$. Logarithmic differentiation of the product formula gives 
\begin{equation}
\label{LoverLExplicit}
    -\frac{L'(s, \chi)}{L(s, \chi)} = \sum_{n=1}^{\infty}\Lambda(n) n^{-\sigma}\chi(n)e^{-i t \log n},
\end{equation}
which is obtained in the same way as for $\zeta(s)$, but with an added factor of $\chi(n)$. Owing to the fact that $\chi(n)$ is a $\phi(q)$-th root of unity (as long as $n$ and $q$ are coprime), we may write $\mathfrak{R}\chi(n)e^{-i t \log n}$ as $\cos\theta$, for some $\theta$. Replacing $\chi$ by $\chi^{2}$ doubles the contribution to the argument from $\chi$ (as it is a root of unity). Therefore $\mathfrak{R}\chi^{2}(n)e^{-2 i t \log n} = \cos 2\theta$. Moreover, replacing $\chi$ by $\chi_0$ and $t$ by zero produces $\cos 0 = 1$, so we may write an analogous inequality to (\ref{LinCombZetaInequality}) involving L-functions: namely
\begin{equation}
\label{LinCombLRelation}
    3\left[-\frac{L'(\sigma, \chi_0)}{L(\sigma, \chi_0)} \right] + 4\left[-\mathfrak{R} \frac{L'(\sigma + i t, \chi)}{L(\sigma + i t, \chi)} \right] + \left[-\mathfrak{R} \frac{L'(\sigma + 2i t, \chi^{2})}{L(\sigma + 2i t, \chi^{2})} \right] \geq 0.
\end{equation}
Notice that if $\chi$ is a real character, $\chi^{2} = \chi_0$, and this can cause serious problems. For now, we shall assume that $\chi$ is complex so that this issue is avoided. Recall (\ref{LPartialFraction}), which implies that 
\begin{equation}
\label{RealLogDiffL}
    -\mathfrak{R}\frac{L'(s, \chi)}{L(s, \chi)} = \frac12 \log\frac{q}{\pi} + \frac12 \mathfrak{R}\frac{\Gamma'(s/2 + a/2)}{\Gamma(s/2 + a/2)} - \mathfrak{R}B(\chi) - \mathfrak{R}\sum_{\rho} \left( \frac{1}{s - \rho} + \frac{1}{\rho} \right). \nonumber
\end{equation}
By logarithmic differentiation of the Hadamard product for $\xi(s, \chi)$, followed by application of the functional equation, we have
\begin{equation}
    B(\chi) = \frac{\xi'(0, \chi)}{\xi(0, \chi)} = -\frac{\xi'(1, \overline{\chi})}{\xi(1, \overline{\chi})} = -B(\overline{\chi}) - \sum_{\overline{\rho}}(\frac{1}{1-\overline{\rho}} + \frac{1}{\overline{\rho}}), \nonumber
\end{equation}
where $\overline{\rho}$ are the non-trivial zeros of $L(s, \overline{\chi})$. Moreover, it is clear that $B(\overline{\chi}) = \overline{B(\chi)}$, so 
\begin{equation}
    2\mathfrak{R}B(\chi) = -\sum_{\overline{\rho}}(\mathfrak{R}\frac{1}{1-\overline{\rho}} + \mathfrak{R}\frac{1}{\overline{\rho}}). \nonumber
\end{equation}
Since the zeros of L-functions are symmetrically distributed around the critical line, we may change $1-\overline{\rho}$ to $\rho$ without changing the sum, as this is simply a permutation of the terms. Therefore
\begin{equation}
    \mathfrak{R}B(\chi) = -\frac12 \sum_{\rho}(\mathfrak{R}\frac{1}{\rho} + \mathfrak{R}\frac{1}{\overline{\rho}}) = -\sum_{\rho}\mathfrak{R}\frac{1}{\rho}. \nonumber
\end{equation}
Substituting this into (\ref{RealLogDiffL}), combined with the fact that the $\Gamma'/\Gamma$ term is $O\left(\log(2 + t)\right)$ as in the previous section, we have
\begin{equation}
\label{LOverLBound}
    -\mathfrak{R}\frac{L'(s, \chi)}{L(s, \chi)} < c_1 \mathcal{L} - \sum_{\rho}\mathfrak{R}\frac{1}{s - \rho}, \quad \textrm{where} \quad  \mathcal{L} \coloneqq \log q + \log (t + 2), 
\end{equation}
which holds for any $\sigma > 1$ and any primitive character $\chi$ (we have not yet used the assumption that $\chi$ is complex). Since
\begin{equation}
    \mathfrak{R}\frac{1}{s-\rho} = \frac{\sigma - \beta}{\abs{s-\rho}^{2}} \geq 0, \quad (\sigma > 1), \nonumber
\end{equation}
we may as before throw away any number of the terms in the sum without changing the inequality. It is not immediately obvious that the inequality (\ref{LOverLBound}) holds for $L(s, \chi^{2})$, since $\chi^{2}$ is non-trivial, but not necessarily primitive. However, suppose $\chi^{2}$ is induced by a character $\chi_1$. Then by (\ref{InducedCharacterRelation}), we have
\begin{equation}
    \log L(s, \chi^{2}) = \log L(s, \chi) + \sum_{p \rvert q} \log(1 - \chi_1(p)p^{-s}), \nonumber
\end{equation}
which implies
\begin{equation}
    \frac{L'(s, \chi^{2})}{L(s, \chi^{2})} = \frac{L'(s, \chi_1)}{L(s, \chi_1)} + \sum_{p \rvert q} \frac{-\chi_1(p)p^{-s}\log p}{1 - \chi_1(p)p^{-s}}. \nonumber
\end{equation}
Therefore
\begin{equation}
    \abs{\frac{L'(s, \chi^{2})}{L(s, \chi^{2})} - \frac{L'(s, \chi_1)}{L(s, \chi_1)}} \leq \sum_{p \rvert q} \frac{p^{-\sigma}\log p}{1 - p^{-\sigma}} \leq \sum_{p \rvert q} \log p \leq \log q. \nonumber
\end{equation}
The first inequality is just the triangle inequality, the second inequality may be shown by the fact that $\abs{x(1-x)^{-1}} \leq 1$ for all $x \in [0, 1/2]$ (just check the derivative), while the third is since the sum of logarithms is the logarithm of the product. Thus, the inequality (\ref{LOverLBound}) also holds for $L'(s, \chi^{2})/L(s, \chi^{2})$. We proceed as in the previous section, omitting the entire series in (\ref{LOverLBound}) for $L'(\sigma + 2it, \chi^{2})/L(\sigma + 2it, \chi^{2})$. We choose $t$ to coincide with the imaginary part of a non-trivial zero of $L(s, \chi)$, and keep only the term in the series corresponding to this zero, which gives
\begin{equation}
    -\mathfrak{R}\frac{L'(\sigma + it, \chi)}{L(\sigma + it, \chi)} < c_1 \mathcal{L} - \frac{1}{\sigma - \beta}. \nonumber
\end{equation}
Finally, the term for the trivial character has the same bound as in the previous section for $\zeta(\sigma)$, say $(\sigma - 1)^{-1} + c_2$. It follows from (\ref{RealLogDiffL}) that
\begin{equation}
    \frac{4}{\sigma - \beta} < \frac{3}{\sigma - 1} + c_3 \mathcal{L}. \nonumber
\end{equation}
Setting $\sigma = 1 + c_4\mathcal{L}^{-1}$, we have as in the previous section (choosing an appropriate $c_4$)
\begin{equation}
\label{InitialLZeroFreeRegion}
    \beta < 1 - \frac{c_5}{\mathcal{L}}
\end{equation}
for an absolute constant $c_5$. This inequality is also true for complex non-primitive characters. Indeed, if the character $\chi$ is induced by a primitive $\chi_1$, then any zeros of $L(s, \chi)$ are either zeros of $L(s, \chi_1)$ or any of the terms $(1 - \chi_1(p)p^{-s})$ in the product part of (\ref{InducedCharacterRelation}). Any zero of those terms must lie on $\sigma = 0$, so the relation (\ref{InitialLZeroFreeRegion}) still holds for the zeros of $L(s, \chi)$. \\

Suppose now that $\chi$ is a real primitive character. We now have the issue that $\chi^{2} = \chi_0$. Note that this is the same as having $\chi^{2}$ induced by the trivial character of modulus 1, so we may refer to our previous argument, which implies that
\begin{equation}
    \abs{\frac{L'(s, \chi_0)}{L(s, \chi_0)} - \frac{\zeta'(s)}{\zeta(s)}} \leq \log q \nonumber
\end{equation}
for $\sigma > 1$. With regard to $\zeta'/\zeta$, we refer to (\ref{ZetaPartialFraction}), which with our usual estimates, and not assuming that $t$ is large, gives
\begin{equation}
    -\mathfrak{R}\frac{\zeta'(s)}{\zeta(s)} < \mathfrak{R}\frac{1}{s-1} + c_6\log t. \nonumber
\end{equation}
Thus, combining the previous two inequalities using the triangle inequality gives
\begin{equation}
    -\mathfrak{R}\frac{L'(\sigma + 2it, \chi_0)}{L(\sigma + 2it, \chi_0)} < \mathfrak{R}\frac{1}{\sigma - 1 + 2 i t} + c_7 \mathcal{L}, \nonumber 
\end{equation}
with $\mathcal{L}$ as before. We therefore replace the inequality used for complex $\chi$ by this one, and by (\ref{LinCombLRelation}) we have
\begin{equation}
    \frac{4}{\sigma - \beta} < \frac{3}{\sigma - 1} + \mathfrak{R}\left(\frac{1}{\sigma - 1 + 2it}\right) + c_8\mathcal{L}, \nonumber
\end{equation}
where in this case $t=\gamma$ so as to coincide with the imaginary part of a non-trivial zero of $L(s, \chi)$. Take $\sigma = 1 + \delta/\mathcal{L}$, and require that $\delta/\mathcal{L} < \gamma$, so that
\begin{equation}
    \frac{4}{\sigma - \beta} < \frac{3\mathcal{L}}{\delta} + \frac{\mathcal{L}}{5\delta} + c_8\mathcal{L}. \nonumber
\end{equation}
Therefore, with some rearranging we have
\begin{equation}
    \beta < 1 - \frac{4 - 5c_8\delta}{16 + 5c_8\delta}\frac{\delta}{\mathcal{L}}. \nonumber
\end{equation}
This gives the same type of bound for a complex character $\chi$, subject to the condition that $\gamma \geq \delta/\mathcal{L}$, which is certainly satisfied when $\gamma \geq \delta/\log q$. As before, $\chi$ need not be primitive. It now remains to consider when $\abs{t} < \delta/\log q$. 
\begin{proposition}
There is at most one zero of $L(s, \chi)$ for a real non-trivial character $\chi$ satisfying $\beta > 1 - \delta/\log q$. 
\end{proposition}
It is a consequence of this proposition that any such zero must be real. Indeed, for a real character $\chi$, we have $\overline{\chi} = \chi$, so zeros of $L(s, \chi)$ are consequently symmetric about the real axis. Thus, existence of a complex zero in this region guarantees the existence of at least two - a contradiction. Therefore, assume that $L(s, \chi)$ has zeros at $\beta \pm i\gamma$ for $\gamma \neq 0$. Now, (\ref{LOverLBound}) may be adapted for the case of $s = \sigma > 1$ to give
\begin{equation}
    -\frac{L'(\sigma, \chi)}{L(\sigma, \chi)} < c_{10} \log q - \sum_{\rho}\frac{1}{\sigma-\rho}, \nonumber
\end{equation}
where the sum is real since each zero occurs in a conjugate pair (or is itself real). Note that this inequality also requires $\chi$ to be primitive. We may assume this, since we may adapt our argument as previously to induced characters. Keeping only the terms corresponding to the zeros $\beta \pm i\gamma$, we have
\begin{equation}
    -\frac{L'(\sigma, \chi)}{L(\sigma, \chi)} < c_{10} \log q - \frac{2(\sigma - \beta)}{(\sigma - \beta)^{2} + \gamma^{2}}. \nonumber
\end{equation}
For the left hand side, we may bound by
\begin{equation}
    -\frac{L'(\sigma, \chi)}{L(\sigma, \chi)} = \sum_{n=1}^{\infty}\chi(n)\Lambda(n)n^{-\sigma} \geq -\sum_{n=1}^{\infty}\Lambda(n)n^{-\sigma} = \frac{\zeta'(\sigma)}{\zeta(\sigma)} > \frac{-1}{\sigma -1} - c_{11}, \nonumber
\end{equation}
so that
\begin{equation}
    -\frac{1}{\sigma - 1} < c_{12}\log q - \frac{2(\sigma - \beta)}{(\sigma - \beta)^{2} + \gamma^{2}}. \nonumber
\end{equation}
Taking $\sigma = 1 + 2\delta/\log q$, we have by assumption on $\gamma$ that
\begin{equation}
    \abs{\gamma} < \frac{\delta}{\log q} = \frac12 (\sigma - 1) < \frac12 (\sigma - \beta). \nonumber
\end{equation}
Substituting into the previous inequality, we have
\begin{equation}
    -\frac{\log q}{2\delta} = -\frac{1}{\sigma - 1} < c_{12}\log q - \frac{8}{5(\sigma - \beta)}. \nonumber
\end{equation}
Rearranging this gives
\begin{equation}
    \beta < 1 - \frac{2(8 - 5(1 + 2\delta c_{12}))}{5(1 + 2\delta c_{12})}\frac{\delta}{\log q}. \nonumber
\end{equation}
Choosing $\delta$ well in relation to $c_{12}$ therefore implies $\beta < 1 - \delta/\log q$ (it is important to note that this is the $\delta$ of the previous result). It suffices therefore to consider the case that we have 2 real zeros (or a double real zero). The argument for this case is essentially the same, with $\gamma = 0$. Therefore, we have proved that the only zero satisfying
\begin{equation}
    \beta > 1 - \delta/\log q, \quad \abs{\gamma} < \frac{\delta}{\log q}, \nonumber
\end{equation}
is a single real zero. In essence, we have now proved an (almost) zero free region for all L-functions, which may be stated in the following theorem. 
\begin{theorem}
There exists a positive constant $c$ such that if $\chi$ is a complex character modulo $q$, there is no zero in the region
\begin{align}
    \sigma \geq \left\{\begin{array}{ll}
         & 1 - c/\log q\abs{t} \quad \textrm{if} \ \abs{t} \geq 1, \\
         & 1 - c/\log q \quad \textrm{if} \ \abs{t} \leq 1.
    \end{array}\right\}. \nonumber
\end{align}
Moreover, if $\chi$ is a real non-trivial character, there is at most one zero in this region, in which case it is a single real zero.
\end{theorem}
The possible existence of real zeros close to $s=1$ is a real issue, and bounding the distance they must lie from $s=1$ is difficult. We shall defer this until later. Once this is proved, the key step in Dirichlet's theorem immediately follows; namely that all L-functions of non-trivial character are non-zero at $s=1$. This is absolutely not the most direct way of proving this fact, but our aim is to prove an even stronger result about primes in arithmetic progressions than Dirichlet's theorem.
\section{The Numbers \texorpdfstring{$N(T)$}{Lg} and \texorpdfstring{$N(T, \chi)$}{Lg}}
We now turn our attention to formulas of the type stated by Riemann and proved by von Mangoldt. These concern the number of zeros of L-functions below a given height in the critical strip - essentially estimating how frequently the non-trivial zeros occur. We first study the function $N(T, \chi)$, defined as the number of zeros of $L(s, \chi)$ in the rectangle $0 < \sigma < 1$, $\abs{t} < T$. It suffices to study the case of $\chi$ primitive and for now we shall assume $q > 1$. We proceed by the principle of argument on the function $\xi(s, \chi)$, over the rectangle $R$ with vertices at 
\begin{equation}
    5/2 - iT, \quad 5/2 + iT, \quad -3/2 + iT, \quad -3/2 - iT, \nonumber
\end{equation}
traversed anti-clockwise. Since $\xi(s, \chi)$ is holomorphic inside this region, its change in argument is precisely the number of zeros of the function (with multiplicity). Moreover, since the only zeros of this function lie at non-trivial zeros of $L(s, \chi)$, we have that
\begin{equation}
\label{ArgumentPrinciple}
    2\pi N(T, \chi) = \Delta_{R}\arg \xi(s, \chi),
\end{equation}
where the right hand represents the total change in argument of $\xi(s, \chi)$ over the rectangle $R$. Split the rectangle into two halves along the line $\sigma = 1/2$, and call the right half $L$. By the functional equation, we have
\begin{equation}
    \arg \xi(\sigma + it, \chi) = c + \arg \xi(1 - \sigma - it, \overline{\chi}) = c + \arg\overline{\xi(1 - \sigma + it, \chi)}, \nonumber
\end{equation}
so that the change in argument along each half of the rectangle is the same ($c$ here is independent of $s$). Thus it suffices to consider $L$. We have
\begin{equation}
\label{ArgumentPiTerm}
    \Delta_{L} \arg \left(\frac{q}{\pi}\right)^{(s + a)/2} = \arg \left(\frac{q}{\pi}\right)^{iT/2} - \arg \left(\frac{q}{\pi}\right)^{-iT/2} = T\log \left(\frac{q}{\pi} \right). \nonumber
\end{equation}
In order to estimate the change in argument of the $\Gamma(s/2 + a/2)$ term, we must use Stirling's formula (Theorem~\ref{StirlingFormula}) which states that
\begin{equation}
    \log \Gamma(s) = (s - \frac{1}{2}) \log s - s + \frac12 \log2\pi + O(\abs{s}^{-1}). \nonumber
\end{equation}
We have that
\begin{equation}
    \Delta_{L} \arg \Gamma\left(\frac{1}{2}(s/2 + a/2)\right) = \mathfrak{I}\log \Gamma\left(\frac{\alpha}{2}\right) - \mathfrak{I}\log \Gamma\left(\overline{\frac{\alpha}{2}}\right), \nonumber
\end{equation}
upon setting $\alpha = \frac12 + a + iT$, by the properties of complex logarithm. Notice that changing $s$ for $\overline{s}$ in Stirling's formula simply gives its complex conjugate (with the same error term), so that
\begin{equation}
    \log \Gamma(z) - \log\Gamma(\overline{z}) = 2i \mathfrak{I} \left( (z - \frac12)\log z - z + \frac12\log2\pi \right) + O(\abs{z}^{-1}). \nonumber 
\end{equation}
Setting $z = \alpha/2$ into this expression, we have
\begin{align}
\label{ArgumentGamma}
    \Delta_{L}\arg \Gamma(s/2 + a/2) &= \mathfrak{I}\left[ (\alpha - 1) \log \frac{\alpha}{2} - \alpha + \log 2\pi \right] + O(1) \nonumber \\
    &= T \log \frac{T}{2} - T + O(1).
\end{align}
It therefore remains to estimate $\Delta_{L}\arg L(s, \chi)$ and we aim to prove that it is $O(\log q T)$. It suffices to consider $\arg L(\frac12 + iT, \chi)$. To do this, we will make use of the following useful lemma. 
\begin{lemma}
If $\rho = \beta + i\gamma$ are the non-trivial zeros of $L(s, \chi)$ where $\chi$ is primitive, then for all real $t$, we have
\begin{equation}
    \sum_{\rho}\frac{1}{1 + (t - \gamma)^{2}} = O(\mathcal{L}) \nonumber
\end{equation}
where $\mathcal{L} \coloneqq \log \left(q(\abs{t} + 2)\right)$.
\end{lemma}
\begin{proof}
Recall (\ref{LOverLBound}) of the previous section, where we had
\begin{equation}
    -\mathfrak{R}\frac{L'(s, \chi)}{L(s, \chi)} < c_1 \mathcal{L} - \sum_{\rho}\mathfrak{R}\frac{1}{s - \rho}. \nonumber
\end{equation}
Set $s = 2 + it$. On this line $\abs{L'/L}$ is bounded by an absolutely convergent sum independent of $t$, so this term may be dispensed of in the equation above to give
\begin{equation}
    \sum_{\rho}\mathfrak{R}\frac{1}{(2 + it) - \rho} \ll \mathcal{L}. \nonumber
\end{equation}
Moreover, we have
\begin{equation}
    \mathfrak{R}\frac{1}{(2 + it) - \rho} = \frac{(2 - \beta)}{(2 - \beta)^{2} + (t - \gamma)^{2}} \gg \frac{1}{1 + (t - \gamma)^{2}}. \nonumber
\end{equation}
In conclusion, 
\begin{equation}
    \sum_{\rho}\frac{1}{1 + (t - \gamma)^{2}} = O(\mathcal{L}). \nonumber
\end{equation}
\end{proof}
Now, two conclusions are immediate from this lemma. First, the number of terms satisfying $\abs{t - \gamma} < 1$ is of order at most $\mathcal{L}$, since if the terms of the sum are all $O(1)$, the number of terms must be at most $O(\mathcal{L})$. Secondly, the sum over zeros with $\abs{t - \gamma} \geq 1$ is also $O(\mathcal{L})$ for the same reason. Now, it follows that 
\begin{equation}
\label{PartialLRestrictedSum}
    \frac{L'(s, \chi)}{L(s, \chi)} = \sum_{\rho: \ \abs{t - \gamma} < 1} \frac{1}{s - \rho} + O(\mathcal{L}).
\end{equation}
To see this, we use the equation (\ref{LPartialFraction}). Setting $s = 2 + it$ and subtracting from the original equation, we have
\begin{equation}
    \frac{L'(s, \chi)}{L(s, \chi)} - \frac{L'(2 + it, \chi)}{L(2 + it, \chi)} = F(s) + \sum_{\rho} \left(\frac{1}{s - \rho} - \frac{1}{2 + it - \rho}\right), \nonumber
\end{equation}
where $F(s)$ is a term involving $\Gamma'/\Gamma$, which by Stirling's formula is essentially $O(\mathcal{L})$. As in the proof of the previous lemma, $L'(2 + it, \chi)/L(2 + it, \chi)$ is $O(1)$, so we have
\begin{equation}
    \frac{L'(s, \chi)}{L(s, \chi)} = \sum_{\rho}\left(\frac{1}{s - \rho} - \frac{1}{2 + it - \rho}\right) + O(\mathcal{L}). \nonumber
\end{equation}
It remains to estimate the sum. First, consider the case when $\abs{t - \gamma} \geq 1$. We have
\begin{equation}
    \abs{s - \rho}^{2} \geq (t - \gamma)^{2}, \quad \abs{2 + it - \rho}^{2} \geq (t - \gamma)^{2}, \nonumber
\end{equation}
and we may take square roots while maintaining the inequality. Thus
\begin{equation}
    \abs{\frac{1}{s - \rho} - \frac{1}{2 + it - \rho}} = \frac{2 - \sigma}{\abs{s - \rho}\abs{2 + it - \rho}} \leq \frac{3}{(t - \gamma)^{2}} \ll \frac{1}{1 + (t - \gamma)^{2}}, \nonumber
\end{equation}
so by the previous lemma, the sum over these terms is $O(\mathcal{L})$. When $\abs{t - \gamma} < 1$, we have $\abs{2 + it - \rho} \geq 1$, and the number of terms in the sum is $O(\mathcal{L})$ as discussed before. Therefore
\begin{equation}
    \sum_{\rho: \abs{t - \gamma} < 1} \left(\frac{1}{s - \rho} - \frac{1}{2 + it - \rho} \right) = \sum_{\rho: \abs{t - \gamma} < 1} \frac{1}{s - \rho} + O(\mathcal{L}). \nonumber
\end{equation}
We therefore put everything together to give (\ref{PartialLRestrictedSum}). We may now estimate $\Delta_{L}\arg L(s, \chi)$. For the vertical line from $5/2 - it$ to $5/2 + it$, by (\ref{LoverLExplicit}) we have
\begin{equation}
    \Delta \arg L(s, \chi) = \mathfrak{I} \int_{5/2 - it}^{5/2 + it} \frac{L'(s, \chi)}{L(s, \chi)}\mathrm{d}s = \mathfrak{I}\left[\sum_{n=1}^{\infty} \frac{\chi(n)\Lambda(n)}{n^{s} \log n} \right]_{5/2 - it}^{5/2 + it} = O(1). \nonumber
\end{equation}
For the remaining line segments, we use (\ref{PartialLRestrictedSum}), giving
\begin{equation}
    \Delta\arg L(s, \chi) = \mathfrak{I}\int_{1/2 \pm it}^{5/2 \pm it} \frac{L'(s, \chi)}{L(s, \chi)}\mathrm{d}s = \sum_{\rho: \abs{t - \gamma} < 1} \mathfrak{I} \int_{1/2 \pm it}^{5/2 \pm it} (s - \rho)^{-1} \mathrm{d}s + O(\mathcal{L}). \nonumber 
\end{equation}
Each integral inside the sum is given by $\Delta \arg (s - \rho)^{-1}$ along each of the segments, which is $O(1)$ as it is bounded by $\pi$. There are also $O(\mathcal{L})$ terms in the sum by the above, so the change in argument of $L'/L$ over these segments is $O(\mathcal{L})$. Putting everything together, we have
\begin{equation}
\label{ArgumentLFunction}
    \Delta_{L} \arg L(s, \chi) = O(\mathcal{L}).
\end{equation}
We now restrict to a fixed $t = T \geq 2$. Then, putting together (\ref{ArgumentPiTerm}), (\ref{ArgumentGamma}), and (\ref{ArgumentLFunction}), we have
\begin{equation}
    \Delta_{L}\arg \xi(s, \chi) = T\log \frac{qT}{2\pi} - T + O(\log qT). \nonumber
\end{equation}
In conclusion, using (\ref{ArgumentPrinciple}), and the fact that the contribution over $R$ is twice that over $L$, we have
\begin{equation}
\label{vonMangoldtLFunction}
    \frac12 N(T, \chi) = \frac{T}{2\pi} \log \frac{q T}{2 \pi} - \frac{T}{2 \pi} + O(\log qT).
\end{equation}
The argument is easily modified for $q = 1$, and the function $N(T)$, defined as the number of zeros of $\zeta(s)$ in the region $0 < \sigma < 1$, $0 \leq t \leq T$. Note that it is sufficient to consider the upper half-plane in this case, as $\zeta(s)$ is symmetric about the real-axis. The only modification required is a multiplicative factor of $s(s - 1)/2$ in the function $\xi(s, \chi)$ so that it is entire, and its change in argument gives only zeros (and not poles). The change of argument of this factor over the right-half of the rectangle is easily calculated to be essentially
\begin{equation}
    \pi + O(T^{-1}), \nonumber
\end{equation}
so can be ignored, as it is lower order than the error term in (\ref{vonMangoldtLFunction}), with which we recover the famous Riemann von-Mangoldt formula, 
\begin{equation}
    N(T) = \frac{T}{2\pi} \log \frac{T}{2 \pi} - \frac{T}{2 \pi} + O(\log T), \nonumber
\end{equation}
for $T \geq 2$. Note that the factor of $1/2$ in the more general formula above is for ease of comparison with this well-established formula for $\zeta(s)$. We now have sufficient information with regard to the non-trivial zeros of L-functions (and in particular $\zeta(s)$) in order to proceed, and prove at least one landmark result regarding the distribution of primes. We now prove formulae such as (\ref{ExpicitPsiFormula}), which expose the key relation of these zeros to the distribution of prime numbers. 

\chapter{The Prime Number Theorem}
\section{The Distribution of the Set of All Primes}
\section{The Prime Number Theorem for Arithmetic Progressions}
\section{Siegel's Theorem}

\chapter{Conclusions}
\appendix
\appendixpage
\addappheadtotoc
\end{document}


